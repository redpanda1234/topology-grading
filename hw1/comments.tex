\documentclass{fkpset}

\name{Forest Kobayashi}
\class{Topology}
\duedate{02/07/2019}
\assignment{HW 1 Comments}

\chead{HW 1 Comments}
\rhead{Math 147 -- Spring 2019}

\titlespacing*{\section}{0pt}{4em plus 1 em minus 2em}{1 em plus 1em minus .4em}
\titlespacing*{\subsection}{0pt}{1.5em}{1 em plus .4 em minus .2em}
\titlespacing*{\subsubsection}{0pt}{1em}{1em plus .4em minus .2em}


\usepackage{hyperref}
\usepackage{fkmisc}

\lstset{language=[LaTeX]TeX,
  % Colors and stuff
  backgroundcolor=\color{white},     % choose the background color
  basicstyle=\footnotesize\ttfamily, % set size + style of font
  %
  commentstyle=\color{mygray},% comment style
  identifierstyle=\color{blue},
  keywordstyle=\color{mygreen},      % keyword style
  stringstyle=\color{orange},        % string literal style
  %
  % Whitespace handling
  breakatwhitespace=false,           %
  breaklines=true,                   % sets automatic line breaking
  keepspaces=true,                   % needs (columns=flexible)?
  showspaces=false,                  % Don't add weird underscores
  showstringspaces=false,            % Same
  showtabs=false,                    % Same
  tabsize=2,	                       % 1 tab = 2 spaces
  xleftmargin=2em,                   % Properly indent listing
  %
  % Caption
  belowcaptionskip=1\baselineskip,   % Spacing to caption
  captionpos=b,                      % caption at bottom
  title=\lstname,                    % show fname w/ \lstinputlisting
  %
  % Framing & Numbering
  frame=L,	                        % adds a frame on the left
  numbers=left,                      % linum loc. (none, left, right)
  numbersep=10pt,                    % distance of linum to code
  numberstyle=\tiny\color{numgray},  % style used for the line-numbers
  rulecolor=\color{black},           %
  stepnumber=1,                      % number every line
  %
  % Misc
  language=[LaTeX]Tex,                   % default language
  % texcsstyle=*\bf\color{blue},
  % otherkeywords={$, \{, \}, \[, \]},
}

\begin{document}
\section{Important Notes:}
Before we get to the main stuff, a few meta-notes about the grading:
\begin{enumerate}[label=(\arabic*)]
  \item \textbf{Important:} If anything I've said here seems to conflict with
    what Prof.\ Su has told you in class, let one of us know so that we can talk
    about it during our weekly grading meeting.
  \item If any of my comments are ever unclear and/or you think I've made a
    mistake, you should contact me directly at
    \href{mailto:fkobayashi@g.hmc.edu}{fkobayashi@g.hmc.edu}, and I'll try to
    get back to you as soon as I can! Note, I probably won't be receive your
    message if you send it via facebook messenger (since I don't use facebook
    often), so don't send things there.
  \item The tentative grading scheme is as follows:
    \begin{itemize}
      \item Each problem is worth 5 points.
      \item 4 points go to the correctness of your argument.
      \item 1 point will go to the clarity of your writing. Since this is likely
        something that wasn't emphasized in your previous courses, points will
        not be deducted here until roughly the fourth problem set.
      \item If you score lower than a 4/5 on any particular problem, you can
        resubmit a rewrite once (but only once) for an opportunity for full
        credit!
      \item If you accidentally forgot to attach a problem and/or you submitted
        a proof of the wrong problem (\texttt{\#relatable}), talk to Prof.\ Su
        first --- I can't award any points here until he gives the OK, but
        chances are he'll let you resubmit to me for credit \texttt{:)}
    \end{itemize}
    The details might change a bit as we test this out, but the main thing to
    note is that the clarity of your writing will be directly factored into your
    grade. See section \ref{sec:rationale} below for the rationale.
  \item Here's how the comments work. Each will consist of an \textbf{A},
    \textbf{B}, or \textbf{C}, followed by some $\mb n \in \NN$. The categories
    are as follows:
    \begin{itemize}
      \item $\mb{An}$ comments are about style/formatting standards for
        mathematical writing. This is something we've decided to emphasize, so
        consider skimming them each week even if you didn't receive a comment!
      \item Labels of the form $\mb{Bn}$ will refer to comments about the
        correctness of your proof.
      \item Labels of the form $\mb{Cn}$ are usually just some comments on
        notational standards / \LaTeX\ tips (if applicable). You might not see
        these every week.
    \end{itemize}
    When grading your homeworks, I tried to stay consistent with the comment
    labeling scheme given above. However, I do make mistakes sometimes, so if
    something looks funny, don't hesitate to send me an email!
  \item This week, I accidentally did all of the grading in pencil, so it might
    be hard to read some of my comments. Sorry about this! In the future, I will
    try to adhere to the following scheme:
    \begin{itemize}
      \item Strikethrough in pencil means that I'm offering a suggested edit to
        your phrasing (per the writing style standards below). It does not
        necessarily mean that your argument is incorrect! Also, usually when I
        do this, I'll offer a reference to a grading comment or give a suggested
        replacement phrasing.
      \item Red pen will be used for comments on correctness.
      \item Anything else we add will be specified in the future.
    \end{itemize}
\end{enumerate}

\vspace{2em}

\section{General comments and standards:}
\subsection{Why the emphasis on writing style?}\label{sec:rationale}
Well, to quote Prof.\ Su's handout on good mathematical writing,
\begin{leftbar}
  ``Learning to communicate effectively is not just a service to your audience;
  it is also an exercise in clarifying and structuring your own thinking.''
\end{leftbar}
This is important! Improving your communication skills doesn't just benefit your
reader, it also benefits \emph{you}. If your proof is very terse and clearly
written, you'll have a much easier time understanding the structure of your
argument, and will also have an easier time identifying potential flaws in your
reasoning. So, without further ado, here are the standards I'll be applying
to your homeworks throughout the semester.

\subsection{Document formatting}
\begin{itemize}
  \item Please leave ample room for comments on your problem sets! This can mean
    larger margins, whitespace after each solution, or both! You don't have to
    put each problem on its own page, but it'd be helpful to have at least
    \verb|\vspace{5cm}| or so \texttt{:)}.
  \item Please do not box your responses to problems! This makes it hard to fit
    my comments in.
  \item Try to leave a little bit of space at the top of the first page of your
    pset in which I can tally points
  \item Only applicable for people who handwrite their psets: Since I'd like to
    use pencil to suggest edits (pencil allows me to write smaller notes and
    also makes it easier for you to read the original version of your proof),
    it'd be awesome if I could ask you to your writeups in pen on white unruled
    paper. If you'd really prefer not to though (e.g., because pencil makes it
    easier to rewrite things), that's totally reasonable! In that case, just try
    to skip every other line so that I have room to write stuff.
\end{itemize}

\subsection{Organizing your reasoning}
TL;DR: you want all of the logic in your proof to be ordered linearly from left
to right, top to bottom. Important structure in your arguments should be
highlighted visually through the use of headers, line breaks, and indentation.

Basically, you want to make it as easy as possible for your reader to understand
your proof on a first reading. Having very linear arguments is helpful in this
respect. However, even if your writing is perfect, the \emph{reader} might make
make a mistake, such as misremembering how the definition of an important
variable. When this occurs, they'll need to go back and re-read an earlier
section, so you want to make it as easy as possible for them to find what
they're looking for. Visual cues are extremely helpful here! And since we're in
the context of homework and psets (instead of the kind of formal mathematical
writing you'd find in a journal), we can choose to employ them!

Now, for the long version:
\subsubsection{Notes on linear arguments:}
As a general rule, any information your reader needs to understand a given
section should be stated as close before where it's needed as possible. Both
``close'' and ``before'' are important here! Try not to structure arguments in
such a way that requires a \texttt{\#throwback} to something from three
paragraphs earlier, and try to \emph{never} cite information that comes later.

These ideas are applicable at all levels of structure in your proof:
\begin{enumerate}[label=(\arabic*)]
  \item Paragraphs: each paragraph should follow logically from the previous
    ones. Paragraphs should never contain a reference to an argument made in a
    \emph{subsequent} paragraph. As an example, suppose we're trying to show $A
    \implies D$. Then we should not do the following:
    \begin{leftbar}\emph{Proof.}
      We have $A$. Thus, [\ldots], and so $B$. As we show in the subsequent
      section, $B \implies C$, and thus we have [\ldots], hence $D$.

      It remains to show $B \implies C$. Suppose $B$, then [\ldots], and so $C$.
      Thus the proof is complete. \qed
    \end{leftbar}
    If the reason you've split up the proof of $B \implies C$ is because the
    argument is long and confusing, then here are two remedies you can try:
    \begin{leftbar}
      \emph{Proof.} First, we prove a small lemma.
      \begin{leftbar}\vspace{.5em}
        \begin{lemma}
          $B \implies C$.
        \end{lemma}
        \emph{Proof of Lemma.} Observe that $B$ implies [\ldots], thus $C$.
        \hfill $\square$
      \end{leftbar}
      \textbf{Main Proof:}
      We have $A$. Thus, [\ldots], and so $B$. By the lemma, $B \implies C$, and
      so $C.$ Then [\ldots], hence $D$. Thus $A \implies D$, as desired. \qed
    \end{leftbar}
    Note, even though we've put the proof of $B \implies C$ earlier on, this
    argument is still linear. Your reader can verify the proof of the lemma on
    its own first, and once they're satisfied it's correct, they can freely
    apply it in the proof body. If you'd like the proof of $B \implies C$'' to
    remain within the main body of your argument, you can frame it as a claim:
    \begin{leftbar}
      We have $A$. Thus [\ldots] $B$.
      \begin{claimproof}
        \item $B \implies C$.
        \item Since $B$, note [\ldots], therefore $C$. \cmark
      \end{claimproof}
      Thus $C$, and so [\ldots], therefore $D$.
    \end{leftbar}
    which is also fine.
  \item Sentences \emph{within} your paragraphs should be arranged so as to
    minimize analogous issues, and
  \item So too with the actual clauses within your sentences. Quick and dirty
    rules:
    \begin{itemize}
      \item Variables should always be defined \emph{before} they're used
      \item Quantification of variables should adhere to the rules described in
        \textbf{A2}
      \item In terms of in-sentence logic, try not to phrase things as ``$B$,
        because $A \implies B$, and we have $A$.'' It's usually better ot say
        something like ``observe that $A$. We have $A \implies B$, so $B$,'' or
        something to that effect.
    \end{itemize}
\end{enumerate}
\subsubsection{Visual cues --- whitespace}
You can actually improve the readability of your arguments substantially just by
making judicious use of whitespace! Some tips:
\begin{enumerate}[label=(\arabic*)]
  \item If the main idea of your proof is $A \implies B \implies C \implies D$,
    then there should be line breaks between your treatment of $A \implies B$,
    $B \implies C$, and $C \implies D$:
    \begin{leftbar}
      \begin{proof}
        Suppose $A$. Then [\ldots], so $B$.

        Now, observe that $B$ implies [\ldots], thus $C$.

        Finally, we want to show $C \implies D$. By definition of $C$, we
        have [\ldots], hence $D$.

        So $A \implies D$, as desired.
      \end{proof}
    \end{leftbar}
    Note, if the proof of each step is only a few lines long, then this isn't
    explicitly necessary. Use your own judgment, and I'll give you feedback if I
    think you should split up your argument more \texttt{:)}
  \item If you introduce a lemma, it might be easier to follow your argument if
    both the statement and proof of the lemma are indented (or otherwise
    visually distinguished) relative to the main proof body. I typically use a
    leftbar environment (sorry the double leftbar looks weird here --- imagine
    the outer one doesn't exist), but you don't have to do this:
    \begin{leftbar}
      \emph{Proof.} First, we introduce a small lemma.
      \begin{leftbar}\vspace{.5em}
        \begin{lemma}
          Hello world! I'm a lemma.
        \end{lemma}
        \emph{Proof of Lemma.} Proof of lemma goes here. Note, I like to use
        $\square$ instead of $\blacksquare$ for the QED symbol for lemmas
        and things, so as to avoid the reader confusing the ``end of lemma
        proof'' with the ``end of main proof.'' \hfill $\square$
      \end{leftbar}
      Now, we proceed to the main proof. Suppose $A$. Then [\ldots], and so
      $B$. By the lemma, $B \implies C$, and [\ldots], thus $D$.
    \end{leftbar}
    A leftbar environment can be defined with the \texttt{mdframed} package
    as follows:
    \begin{lstlisting}
\newmdenv[
  skipabove=5, % whitespace before
  skipbelow=5, % whitespace after
  innerleftmargin = .5em, % indentation of text relative to the bar itself
  innerrightmargin = 0pt, % we don't need to shorten the RHS
  innertopmargin = .5em,  % vertical space between top of leftbar and text
  innerbottommargin = .5em, % ibid but buttom of leftbar
  leftmargin = 2em, % indentation of environment relative to the text
  rightmargin = 0em,
  linewidth = 2pt, % width of the bar; use pt to make independent of font
  topline = false, % no frame at top
  rightline = false, % no frame at right
  bottomline = false % no frame at bottom
  ]{leftbar}\end{lstlisting}\vspace{-3em}
    And you can use the leftbar environment like so:
    \begin{lstlisting}
\begin{leftbar}
  This text is in a leftbar environment
\end{leftbar}\end{lstlisting}\vspace{-3em}
\end{enumerate}
\subsubsection{Visual cues --- common proof types}
For multi-part proofs, it's super helpful to use visual cues instead of verbal
ones to indicate what each component is. I'll give examples below, and TeX code
for each one. For each of the following, note that using \cmark is optional,
\emph{unless} you're making all of your proofs without the use of these headers
(discouraged).
\begin{enumerate}[label=(\arabic*)]
\item To show $A \iff B$, you should format your proof as follows:
  \begin{leftbar}
    \emph{Proof.}
    \begin{iffproof}
      \item Suppose $A$. Then [\ldots lots of math happening here!\ldots], hence
        $B$. \cmark
      \item Suppose $B$. Then [\ldots also lots of math happening here!\ldots],
        therefore $A$.
    \end{iffproof}
    Thus, $A \iff B$. \qed
  \end{leftbar}
  Since syntactically, the ``forward'' direction is technically $B \implies A$
  (``$A$ if $B$''), you should be sure to include a ``suppose $A$'' after the
  $(\implies)$!

  For our purposes, this is easier to read than something like this:
  \begin{leftbar}
    \emph{Proof.} First, we show the forward direction. Suppose $A$. Then
    [\ldots], hence $B$. Now, we show the backward direction. Suppose $B$.
    Then [\ldots] $A$. Since we have $A \implies B$ and $B \implies A$, it
    follows that $A \iff B$.
  \end{leftbar}
  For those of you using \LaTeX, to make the first proof, I define the
  following environment (I believe this requires the \texttt{enumitem}
  package):
  \begin{lstlisting}
% The indentation for customized iff labels really grinds my gears.
% Hence, a new environment to make everything right in the world
% again.
\newcommand*{\iffenum}[1]{\expandafter\@iffenum\csname c@#1\endcsname}
\newcommand*{\@iffenum}[1]{%
  \ifcase#1\or{$(\Rightarrow):$}\or{$(\Leftarrow):$}%
  \else\@ctrerr\fi}

\AddEnumerateCounter{\iffenum}{\@iffenum}{$(\Rightarrow):$}
\newenvironment{iffproof}{%
  \begin{enumerate}[label=\iffenum*, leftmargin=4em]%
  }{\end{enumerate}}\end{lstlisting}\vspace{-3em}
  With this, creating the iff portion of the proof above was as simple as
  \begin{lstlisting}
\begin{iffproof}
  \item Suppose $A$. Then [\ldots lots of math happening here!\ \ldots], hence
    $B$. \cmark
  \item Suppose $B$. Then [\ldots also lots of math happening here! \ldots],
    therefore $A$.
\end{iffproof}\end{lstlisting}\vspace{-3em}
  Where \verb|\cmark| requires the \texttt{pifont} package and is defined
  by\\
  \verb|\newcommand{\cmark}{\text{\ding{51}}}|.
\item Proofs of set equality are almost exactly the same. Let $A$ and $B$ be
  sets. Then you should format your proof like
  \begin{leftbar}
    \emph{Proof.}
    \begin{seteqproof}
    \item Let $x \in A$ be given. Then [\ldots] $x \in B$, so $A \subset B$.
      \cmark
    \item Let $x \in B$ be given. Then [\ldots] $x \in A$, so $B \subset A$.
      \cmark
    \end{seteqproof}
    thus $A = B$. \qed
  \end{leftbar}
  Similarly to the \texttt{iffproof} environment, I have a
  \texttt{seteqproof} environment defined as follows (feel free to set the
  environment name to something more ergonomic, though):
  \begin{lstlisting}
% Similarly, but for subset supset proofs
\newcommand*{\seteqenum}[1]{\expandafter\@seteqenum\csname c@#1\endcsname}
\newcommand*{\@seteqenum}[1]{%
  \ifcase#1\or {$(\subset):$}\or {$(\supset):$}\else\@ctrerr\fi
}
\AddEnumerateCounter{\seteqenum}{\@seteqenum}{$(\subset):$}
\newenvironment{seteqproof}{%
  \begin{enumerate}[label=\seteqenum*, leftmargin=4em]%
  }{\end{enumerate}}\end{lstlisting}\vspace{-3em}
\item When doing a proof with casework, you should format this as follows:
  \begin{leftbar}
    Suppose $A$. We want to show $B$. We have the following cases:
    \begin{enumerate}[label=(\arabic*)]
    \item Suppose (case 1). Then [\ldots] $B$. \cmark
    \item Suppose (case 2). Then we have the following subcases:
      \begin{enumerate}[label=\roman*)]
      \item Suppose (subcase i). Then [\ldots] $B$. \cmark
      \item Suppose (subcase ii). Then [\ldots] $B$. \cmark
      \end{enumerate}
      Since these cases are exhaustive, $B$. \cmark
    \item Suppose (case 3). Then [\ldots] $B$. \cmark
    \end{enumerate}
    Since these cases are exhaustive we have $A \implies B$, as desired.
    \qed
  \end{leftbar}
  The \LaTeX\ code used above again requires the \texttt{enumitem} package,
  and is as follows:
  \begin{lstlisting}
\begin{enumerate}[label=(\arabic*)]
  \item Suppose (case 1). Then [\ldots] $B$. \cmark
  \item Suppose (case 2). Then we have the following subcases:
    \begin{enumerate}[label=\roman*)]
      \item Suppose (subcase i). Then [\ldots] $B$. \cmark
      \item Suppose (subcase ii). Then [\ldots] $B$. \cmark
    \end{enumerate}
    Since these cases are exhaustive, $B$. \cmark
  \item Suppose (case 3). Then [\ldots] $B$. \cmark
\end{enumerate}\end{lstlisting}\vspace{-3em}
  By contrast, you should not do the following:
  \begin{leftbar}
    Suppose $A$. We want to show $B$. First, if (case 1), then [\ldots] $B$. $B$
    also follows from (case 2), because if (subcase i), then [\ldots] and so
    $B$, and if (subcase ii), then [\ldots] and so $B$ as well, so in either
    case, $B$. If (case 3), [\ldots] $B$. So no matter what, $B$. Therefore $A
    \implies B$.
  \end{leftbar}
  A few more points: when doing proofs by casework, you should
  \emph{almost always try to make your cases disjoint.} That is, (case 1)
  should imply ((not case 2) and (not case 3)), and so on. In the rare event
  that this is not the most natural way to write your proof, then you
  should \emph{take special care to highlight to your reader that your cases are
    non-disjoint} --- and be sure to add in an extra sentence at the end
  justifying why they're exhaustive!

  % The exception here is if you have a simple claim like ``prove property $P$ for
  % all points $p \in (A \cup B)$.'' In that case, it's ok

  Also, if you have a situation where (case 2) implies some claim $C$, you
  should \emph{not} start your treatment of (case 2) with ``Suppose $C$.''
  You should start with ``Suppose (case 2). Then $C$, and so [\ldots] thus
  $B$.'' This makes it significantly easier for your reader to follow!
  \item Inductive proofs should look something like
    \begin{leftbar}
      \emph{Proof.} We proceed by induction.

      \textbf{Base Case:} Let $k = k_0$ (where $k_0 \in \NN$ stands for the base
      case here in this example). Observe that [\ldots], thus (statement $P$)
      holds for $k = k_0$. \cmark

      \textbf{Inductive Hypothesis:} Now, suppose that (statement $P$) holds for
      some $k \in \NN$ with $k \geq k_0$.

      \textbf{Inductive Step:} We want to show (statement $P$) holds for $k+1$.
      Note that [\ldots], hence (statement $P$) holds for $k+1$. \cmark

      Thus, by the principle of mathematical induction, (statement $P$) holds
      for all $k \in \NN \st k \geq k_0$.
    \end{leftbar}
    although you have more leeway here.
  \item Proofs by contradiction should \emph{always} start with an explicit
    ``suppose, to obtain a contradiction, that $\neg B$.''
\end{enumerate}
\subsection{Terseness is next to godliness}
Ok, so this might seem hypocritical of me after throwing multiple pages of
formatting guidelines at you, but you should strive to make your writing terse!
In general, try to give your reader \emph{exactly} as much information as they
need --- no more, no less. Sometimes, this means omitting the details for
non-essential steps, and focusing instead on writing in-depth about the
non-trivial portions of your proof. Other times, it means rewriting a long
constructive proof if you realize contradiction is more efficient. For some
perspective, each of the problems from this week were possible to prove in under
$\approx$10 lines, while staying perfectly rigorous!

Here are some DOs and DONTs (this isn't very comprehensive currently; I'll
update it more next week. For the time being, please re-read section 2 of Prof.\
Su's ``good mathematical writing'' document, particularly the part on deciding
what's important to say, avoiding red herrings, stepping back to simplify, and
refining repeatedly).

\textbf{DOs}
\begin{enumerate}[label=(\arabic*)]
  \item Try to remove as much redundancy as possible from your writing! That is,
    don't repeat yourself, and avoid saying the same thing multiple times in a
    row just with slightly different wording. For an example of what \emph{not}
    to do, look at the previous sentence (heh)
  \item Define variables if you find yourself referring the same quantity more
    than once or twice. E.g., if you're doing something involving limit points,
    you'll probably talk a lot about $(U - \set{p}) \cap A$. As such, you might
    consider saying ``let $Y = (U - \set{p}) \cap A$'' so that you don't have to
    rewrite the expression repeatedly and/or make significant use of ``the
    aforementioned intersection.''
  \item Ok, this one requires a bit of qualification, but \emph{do} feel free to
    use symbols and shorthand in your writing \emph{IF} doing so is clearer and
    more efficient than English.

    Ok, yes, I know we've all been taught to avoid shorthand like the plague ---
    but it's not \emph{all} bad if used responsibly! The issue is more that
    often when people start using symbols, they start \emph{overusing} symbols.
    Still, it exists for a reason, so don't discard its usefulness completely.

    There are some situations in which I think shorthand is actually universally
    superior to English. For instance, I believe you should almost always use
    ``$\in$'' instead of ``contained in,'' particulary since ``contained in'' is
    easy to confuse with $\subset$. As an example, I encountered some sentences
    like this when grading:
    \begin{leftbar}
      Every non-empty set in the indiscrete topology is open.
    \end{leftbar}
    On a first reading, it's not clear that the author means ``let $\ms T$ be
    the indiscrete topology on a set $X$. Then every element of $\ms T$ is an
    open set.'' Instead, it sounds like they're saying ``under the indiscrete
    topology, every $U \subseteq X$ is open,'' which is generally false. Thus,
    shorthand and symbols sometimes patch the ambiguities left by English
    language.

    Here's a (non-exhaustive) list of notation I think you should usually use
    over their English counterparts:
    \begin{itemize}
      \item $\abs{X}$ (instead of ``cardinality of $X$'' --- definitely don't
        say ``the size of $X$.'')
      \item $\in, \not\in$
      \item $\subset, \supset, \not\subset, \not\supset$
      \item $<, \leq, =, \geq, >$, and $\neq$
      \item $\cap$ (often $\cup$ as well, but sometimes it's easier to say
        ``together with'' so that you don't have to define extra sets, as in the
        definition of closure)
      \item Set complement (whichever of $A^c$, $X-A$ you prefer --- each has
        its merits; $A^c$ is often easier to read, but leaves ambiguity as to
        what the space we're complementing within is)
      \item Arithmetic operations and stuff
      \item Homeomorphism and equivalence relations ($\cong, \sim, \equiv$)
    \end{itemize}
    Sometimes, the decision for whether or not you should use shorthand depends
    on whether you want your reader to slow down when parsing the current
    section of your proof. E.g., if you're just trying to give the reader a
    summary of what to expect from the following paragraph, it might be ok to
    say ``we want to show $\forall x \in U$, there exists $U_x \in \ms T \st x
    \in U_x$'' before you actually dive into proving it. If, however, you're at
    a really subtle point in the argument, you might consider writing things out
    with more English to force your reader to slow down. Here is a (again,
    non-exhaustive list) of symbols that fall under this ``sometimes'' category:
    \begin{itemize}
      \item $\exists, \forall$
      \item $\implies, \impliedby, \iff$ (although honestly I kind of dislike
        $\impliedby$)
      \item WTS (for ``want to show'' --- be aware, most people intensely
        dislike this), and st or s.t.\ (for ``such that'')
      \item $\contra$ (contradiction)
      \item $\exists!$ (use this one sparingly).
    \end{itemize}
    If you do choose to make use of the symbols above, make sure you don't
    \emph{always} use them. Again, they should be primarily to modulate my
    reading speed. If you want to try out using them, I'll let you know if I
    think it's getting excessive.

    Lastly here're some things you should really try to avoid:
    \begin{itemize}
      \item Most formal logic symbols (e.g., $\land, \lor, \neg, \therefore,
        \because$)
      \item Not much else comes to mind right now, but I'll add more in the
        future if this changes.
    \end{itemize}
\end{enumerate}

\textbf{DONTs}
\begin{enumerate}[label=(\arabic*)]
  \item Don't feel like you need to include non-essential steps, especially if
    they distract from the overall structure of your argument. Taking an example
    from this pset, I would actually encourage you \emph{not} to bother with
    proving that the union of two finite sets is finite. In the future, you can
    simply assert it!

    Remember your audience: 131 is a prerequisite for this class, so you should
    assume that the reader can fill in some of these gaps. If you have something
    that feels like it's in an awkward middle ground, when in doubt, leave a
    proof in a footnote / endnote!
  \item Similarly, don't feel that it's necessary to re-prove results from
    previous classes unless the problem explicitly requires it. As an example,
    since most people have taken Analysis (I think), you can simply assert that
    open balls are open in the standard topology on $\RR^n$. If you'd like to be
    thorough, you can leave a citation for a theorem in a parenthetical, or
    again in a footnote. If you just want the extra practice though, you can
    include a proof as a lemma or something!
  \item Speak with a lot of double negatives. Instead of saying ``$V$ can't
    possibly contain a point $U$ doesn't,'' say ``Every point of $V$ is an
    element of $U$,'' or better yet, ``for all $x \in V$, $x \in U$.''
  \item Don't include red herrings in your proofs! I.e., details / variables
    that you introduce but never use.
  \item Don't include more than one proof of the same fact! If you have a second
    proof that's less intuitive but that you're really excited about, feel free
    to include it as a footnote or after the end of your first proof. But you
    should not put both of them in the main body.
  \end{enumerate}
Ok, onwards to the grading comments.

\section{Writing Style Comments}
\begin{problem}[A1]
  This comment is about proper meta-discourse. When trying to signal to your
  reader what strategies you'll pursue for your proof, you should keep the
  following considerations in mind:\\
  \begin{adjustwidth}{0pt}{1em}
    \begin{itemize}
      \item You don't need to say things things like ``the way to show $A = B$
        is to show $A \subset B$ and $B \subset A$.'' Similarly with ``to prove
        $A \iff B$, we must prove $A \implies B$ and $B \implies A$.'' While
        these are great if writing for a less experienced audience, 147 assumes
        at least two prior proof-based courses, so your reader should be
        familiar with these tactics \texttt{:)}\\

        So, provided it's clear \emph{what} you're trying to prove in these
        cases, you should just use the formatting guidelines listed in the
        \textbf{Visual cues} section.\\
      \item Also, when doing proof by casework, you should use the formatting
        guidelines listed in \textbf{Visual cues}. As stated there, you should
        always try to make it \emph{as clear as possible} that your cases are
        exhaustive. Don't do a proof by casework inline, unless it's really
        short.\\
      \item Before beginning a proof by contradiction, remember to be sure to
        indicate to your reader that that's what you're doing by including a
        ``Suppose, to obtain a contradiction.''\\
    \end{itemize}
  \end{adjustwidth}
\end{problem}
\begin{problem}[A2]
  Properly quantifying variables is extremely important. Here are the rules
  (loosely adapted from Prof.\ Flapan's guidelines --- pay \emph{particular}
  attention to the scoping rules):\vspace{1em}
  \begin{enumerate}[label=(\arabic*)]
    \item Variables defined in the problem statement do not need to be
      re-defined. E.g., if the problem statement says ``prove property $P$ for a
      topological space $(X, \ms T)$,'' you do not need to start your proof with
      ``Let $(X,\ms T)$ be a topological space.''\vspace{.5em}
    \item Other than that, you should \emph{always} define a variable before you
      use it. Variable definitions should come at the start of the sentence, and
      you should give a domain explicitly. E.g., for some property $P$ satisfied
      over a set $X$, say ``$\forall x \in X$, $P(x)$'' instead of ``$P(x)
      \forall x \in X$'' (of course, there are some exceptions --- but this is a
      good starting point). We should \emph{never} say ``we have $P(x)$ for all
      $x$'' without saying what $x$ is.\vspace{.5em}
    % \item If a statement involves multiple variables, then the variables must be
    %   defined in the order they occur in the statment.\vspace{.5em}
    \item ``Let'' declares a variable until the objects the definition depends
      on go out of scope, or until the variable gets redefined. Hence, if you
      start a proof about a topological space $(X,\ms T)$ with ``Let $x \in
      X$,'' then you can freely refer to $x$ until the end of the problem,
      unless ($X, \ms T$) gets redefined somewhere in between. Similarly when
      defining variables in casework.\vspace{.5em}
    \item ``For all,'' ``$\forall$,'' and ``if'' only declare a variable until
      the end of the sentence. If you want to use the variable later, you should
      use ``let.''\\

      This might not be a canonical rule, but we're going to do things this way
      in 147 so that you can use $\forall$ declarations when you need an
      unimportant dummy variable, e.g.\ if you're trying to highlight a property
      of a set using its elements, but don't need to use these element
      later.\vspace{.5em}
    \item Given a set $X$, if you want to prove a claim for all $x \in X$ then
      you must start your proof with ``let $x \in X$ be given'' or ``let $x \in
      X$ be arbitrary.''\vspace{.5em}
    \item ``There exists'' and ``$\exists$'' declare a variable if it does not
      depend on a $\forall$ or an ``if'' statement earlier in the sentence.
      Example:
      \begin{leftbar}
        Consider $\ZZ$. $\forall n \in \ZZ, \exists k\in \ZZ \st k = 2n$. Now,
        take $u = 3+k$.
      \end{leftbar}
      what is $u$ supposed to be? It's unclear. The way we've defined it makes
      it look like a single value, but really it's referring to a whole class of
      values contingent on the particular choice of $n$. By contrast, saying
      \begin{leftbar}
        Let $n \in \ZZ$ be given. Then $\exists k \in \ZZ \st k = 2n$. Now, take
        $u = 3 +k$.
      \end{leftbar}
      is fine, because we know exactly what $u$ is supposed to be, since we're
      assuming we've been given $n$.\\

      The one exception I'll make is for things like
      \begin{leftbar}
        Let $(X,\ms T)$ be a topological space, and suppose $U$ is open. Then
        $\forall x \in X$, there exists $U_x \in \ms T$ such that $x \in U_x$,
        and $U_x \subset X$. Observe that
        \[
          \bigcup_{x \in X} U_x = U,
        \]
        hence $U$ is open.
      \end{leftbar}
      because we're implicitly defining a choice function here, so things work
      out.
  \end{enumerate}
\end{problem}
\begin{problem}[A3]
  Try to avoid giving red herrings and/or unnecessary levels of detail. As an
  example, I saw a lot of people saying things like
  \begin{leftbar}
    ``Let $\lambda$ be a (possibly uncountable) indexing set, and let
    $\set{U_\alpha}_{\alpha \in \lambda}$ be a collection of open sets indexed
    by $\lambda$.''
  \end{leftbar}
  Here, I think it's less confusing to say ``Let $\lambda$ be an arbitrary
  indexing set'' instead of ``let $\lambda$ be a (possibly uncountable) indexing
  set.'' What's the difference? Formally, not much --- ``possibly uncountable''
  is equivalent to ``countable or uncountable,'' which is the same as
  ``arbitrary.'' But this is exactly the problem!\\

  If you say ``possibly uncountable'' instead of ``arbitrary,'' your reader
  might think ``well, \emph{possibly uncountable} is equivalent to
  \emph{arbitrary}, so the only reason the author would pick the former over the
  latter would be if it's going to be important later that $\lambda$ might not
  be countable.'' Then, if no such issue arises, they'll be confused! Hence,
  only choose the ``possibly uncountable'' phrasing if it's really relevant.
  This was a particular example, but try to apply similar reasoning to all of
  your proofs!

  Of course ultimately, what is ``necessary'' vs.\ ``unnecessary'' to show can
  be subjective, so just follow your gut and I'll give you feedback on whether
  or not I agree \texttt{:)}

  % This one is a lot more nitpicky, but in a similar vein, I saw a lot of people
  % saying things like
  % \begin{leftbar}
  %   Note that $3/2 \in (1,2)$, and $3/2 \in \RR$, hence $\RR \cap (1,2) \neq
  %   \varnothing$.
  % \end{leftbar}
  % I would argue this could also be a bit misleading, as it almost sounds like
  % the author is implying that there's something privileged about $3/2$ that
  % allows them to assert both $3/2 \in (1,2)$ and $3/2 \in \RR$. But really, this
  % isn't the case --- they're just as at liberty to claim that \emph{any}
  % arbitrary $x \in (1,2)$ is also in $\RR$. But if that's the case, why not just
  % go straight to saying ``$(1,2) \cap \RR = (1,2) \neq \varnothing$,'' or even
  % ``$(1,2) \cap \RR \neq \varnothing?$'' To me, both of these feel equivalently
  % direct, and both cut out steps.\\


\end{problem}
\begin{problem}[A4]
  Your writing should never include phrases like ``obviously,'' ``clearly,'' or
  ``trivially.'' The rationale is that these words offer no positive content to
  your reader. Consider the following scenarios:
  \begin{enumerate}[label=(\arabic*)]
    \item Suppose that you're saying ``clearly, $A \implies B$'' without
      actually proving it. Then
      \begin{itemize}
        \item In the event that your reader agrees this fact \emph{is}
          self-evident, then you didn't need to say ``clearly'' at all the first
          place.
        \item If your reader does \emph{not} think this fact is self-evident,
          then you risk either making them think you're being lazy, or (if your
          reader is looking to your writing for guidance) maybe even making them
          feel inadequate.
      \end{itemize}
      Neither of these is beneficial, so consider using one of the following
      strategies instead:
      \begin{itemize}
        \item If the claim is sufficiently besides the point, you can simply
          assert it. E.g., ``$A$ is infinite, while $B$ is finite, hence $A \not
          \subset B$.'' You don't have to prove it, but you still shouldn't say
          ``clearly $A \not \subset B$.''
        \item If it seems appropriate, you could give a brief hint of how one
          might show the claim, e.g.\ ``one can use induction to show $A
          \implies B$.'' This gives your reader an idea of what direction to
          look in, and also doesn't sound as abrasive.
        \end{itemize}
    \item Suppose that you're saying ``clearly'' to make a conclusion (e.g.,
      ``Note that [\ldots] thus, we clearly have $A$''). In this case as well,
      the word adds no positive content --- either your preceding argument
      was well-written and sound (in which case the claim should be clear
      independent of whether you assert it is), or it wasn't, in which case
      saying ``clearly'' doesn't actually change anything.
  \end{enumerate}
  In summary, use neutral words like ``observe'' or ``note'' instead of
  ``clearly.''
\end{problem}
\begin{problem}[A5]
  If you find yourself repeatedly saying phrases like ``some point in'' or ``the
  same intersection as before'' to refer to the same object, you should check to
  if your proof reads better if you define this quantity as a variable. Examples:
  \begin{leftbar}
    Suppose $(A \cap B) \cup C \neq \varnothing$. Then there is some point that
    is in $(A\cap B)$ or $C$. It follows that there is some point not contained
    in the complement of this set above, that is, there is some point in $X -
    ((A\cap B)\cup C).$
  \end{leftbar}
  It's often easier for a reader to follow something like this:
  \begin{leftbar}
    Let $Y = (A \cap B) \cup C$, and suppose $Y \neq \varnothing$. Then there
    exists $y \in Y$, and $y \in (A \cap B)$ or $y \in C$.

    $y \not\in X - Y$.
  \end{leftbar}
  It's also shorter!
\end{problem}
\begin{problem}[A6]
  Try not to be redundant! Writing things like
  \begin{leftbar}
    ``Consider $(A \cap B)^C$, that is, the complement of the intersection of
    $U$ and $V$''
  \end{leftbar}
  can often clutter your proofs unnecessarily. Keeping it simple will make it
  easier to read!
\end{problem}
\begin{problem}[A7]
  When trying to say $A \implies B$, $A$, hence $B$, don't say ``$B$, because $A
  \implies B$, and $A$.'' Instead, say something like ``$A$, and $A \implies B$,
  hence $B$.'' If you're concerned that the direction you're going in seems
  unmotivated unless the reader knows what you're trying to do, you can try
  adding a WTS: ``We want to show $B$. We have $A$. But note, $A \implies B$,
  hence $B$.''\\

  In general, justification should come \emph{before} your result. An exception
  is when you're making a parenthetical citation of a theorem --- in this case,
  it's ok to say something like ``$(A \cap B)^c = A^c \cup B^c$ (DeMorgan's
  Laws),'' instead of ``by DeMorgan's Laws, $(A \cap B)^c = A^c \cup B^c.$''
  However, the second form is usually still preferred.
\end{problem}
\begin{problem}[A8]
  Don't use English in situations where symbols are more efficient and
  unambiguous. I saw lots of proofs that said things like
  \begin{leftbar}
    ``The complement of the intersection of the two sets $U$ and $V$ is
    equivalent to the set that is the complement of the union of $U$ and $V$,
    which is true by DeMorgan's Laws,'' or
  \end{leftbar}
  \begin{leftbar}
    ``Since points either live in the set $X-A$ or in the set $A$, then
    because every neighborhood is not a subset of $X-A$, there must be a point
    contained in $A$ in every neighborhood of $p$, by properties of sets.''
  \end{leftbar}
  These are completely correct! However, phrasing it in this way could make it
  harder for a reader to follow what you're saying. The \emph{length} of the
  sentence makes it easy to get lost partway through and forget what's going on,
  especially these come in the middle of a larger proof. By contrast, things
  like\vspace{.5em}
  \begin{itemize}
    \item ``By DeMorgan's Laws, $X - (U \cap V) = (X-U) \cup (X-V)$,'' or
      \vspace{.5em}
    \item ``Observe that $(U \cap V)^C = U^C \cup V^C$ (DeMorgan's Laws)''
  \end{itemize}
  \vspace{.5em}
  are actually much easier for your reader to parse. See the DOs and DONTs
  stuff in \S 2.4 for more!
\end{problem}
\begin{problem}[A9]
  Here are some general comments about vocabulary in math writing:
  \begin{enumerate}
    \item Avoid writing very forcefully (proof by intimidation is not allowed)!
      That is, try not to say things in such a way that it sounds like the
      reader is being pressured into believing you. Examples:\vspace{.5em}
      \begin{enumerate}[label=\roman*)]
        \item Everything in \textbf{A4}\vspace{.5em}
        \item ``We {\color{red}must} have [\ldots]'' and ``it {\color{red}must}
          be [\ldots]''\vspace{.5em}
        \item ``It is a {\color{red} fact} that [\ldots]''\vspace{.5em}
        \item ``{\color{red} Surely}, it follows that [\ldots]''\vspace{.5em}
        \item ``We {\color{red} will} show that [\ldots]''\vspace{.5em}
        \item ``We {\color{red}know} [\ldots]''\vspace{.5em}
        \item ``Thus, {\color{red} we have shown that} [\ldots]''\vspace{.5em}
      \end{enumerate}
      One of the great things about math is that, as a reader, you are never
      under any obligation to believe the writer. At every step of the argument,
      you are equipped to fact-check them, construct counterexamples, and
      generally be a nuiscance. And as a writer, you want to encourage this!
      Remember, your goal is to \emph{be} right, not to sound right. Hence, you
      don't want to use phrasing that makes the reader feel like they're being
      biased. Use neutral, objective phrasing, like
      \begin{enumerate}[label=\roman*)]
        \item See \textbf{A4}
        \item ``{\color{blue}Observe that} [\ldots]''\vspace{.5em}
        \item ``{\color{blue}By definition}, [\ldots]''\vspace{.5em}
        \item ``It follows that [\ldots]''\vspace{.5em}
        \item ``We {\color{blue} want to} show that [\ldots]''\vspace{.5em}
        \item ``We {\color{blue} have} [\ldots]''\vspace{.5em}
        \item ``Thus, [\ldots]''\vspace{.5em}
      \end{enumerate}
      and strive to construct a proof such that by the end, your reader will be
      totally convinced of your correctness, all of their own accord.
    \item This might be a bit too nitpicky, but be cognizant of what different
      words connote.
      \begin{itemize}
        \item ``Assume'' vs.\ ``suppose:'' this might just be me, but in my
          eyes, these words connote two very different things. In my experience
          ``assume'' is usually used when you are imposing a simplifying
          constraint on a problem --- e.g., ``assume $\sin(\theta) = \theta$''
          --- while ``suppose'' is used for proving statements of the form ``if
          $A$ then $B$.''\\

          I think the difference is really one of hypotheticals. To me
          ``assume'' makes it sound like we're saying we're fixing a concrete
          interpretation to the predicate $A$, and saying ``$A$ is true,
          [\ldots], thus $B$ is also true,'' instead of ``\emph{in the event}
          that $A$ is true, it would also follow that $B$ is true.'' Thus I
          prefer ``suppose'' for suppositions.\\
        \item To connect steps in an argument, I try to sample from the
          following pool:
          \begin{itemize}
            \item Inference / premise introduction: ``observe,'' ``note,''
              ``notice,'' ``we have,''
            \item Implications: ``thus,'' ``it follows,'' ``hence,'' ``and so,''
              ``then,'' ``whereby,'' ``observe,''  ``note,'' ``and so,''
              ``whence,'' etc.
            \item Implied by: ``since,'' ``because''
            \item Introducing an unexpected step, or achieving a contradiction:
              ``but,'' ``however''
            \item Final step: add ``therefore,'' to the list under
              ``implications'' (although feel free to break this guideline and
              use ``therefore'' wherever you please)
          \end{itemize}
          And don't really use much else. Notice how few ``implied by'' vocab
          words are listed --- see \textbf{A7}!
      \end{itemize}
  \end{enumerate}
  % General list of mathematical vocabulary dos and don'ts:
  % Also, on usage of ``must'' (to be used sparingly). Use
  % ``therefore'' when concluding larger portions of a proof, and use ``however''
  % when you're doing things like introducing a contradiction. Also, ``assume''
  % should only be used to list premises for constraints in problems. Suppose
  % should be used for suppositions.
  \begin{itemize}
    \item Don't say things like ``The set $A$ has a limit point''
  \end{itemize}
\end{problem}
\section{Correctness}
\subsection{Execercise 3.5}~
\begin{problem}[B1]
  When verifying the finite complement topology, note that $U \in \ms T$ need
  not imply $X - U$ is finite --- we could have $U = \varnothing$, whence $X - U
  = X$.
\end{problem}
% \subsection{Exercise 3.13}~
% \begin{problem}[B2]
%   When showing that
% \end{problem}

\section{Comments on TeX / Notation}
\begin{problem}[C1]
  You might consider using $\ms T$ instead of $\mc T$ to refer to a topology, to
  be consistent with Prof.\ Su's book:
  \begin{lstlisting}
% Use a better script font
\usepackage[mathscr]{euscript}
\newcommand*{\ms}[1]{\ensuremath{\mathscr{#1}}}\end{lstlisting}\vspace{-3em}
  which you can then call simply by \texttt{\ms T}.
\end{problem}
\begin{problem}[C2]
  I saw a lot of people using $B(p, r)$ (or was it $B(r,p)$?) to denote ``the
  ball of radius $\epsilon$ centered about point $p$.'' As far as I can tell,
  this is non-standard notation; I think most people use $B_r(p)$ instead.
\end{problem}
\end{document}