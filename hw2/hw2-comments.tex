\documentclass{fkpset}

\name{Forest Kobayashi}
\class{Topology}
\duedate{02/16/2019}
\assignment{HW 2 Comments}

\chead{HW 2 Comments}
\rhead{Math 147 -- Spring 2019}

\usepackage{hyperref}
\usepackage{fkmisc}

\lstset{language=[LaTeX]TeX,
  % Colors and stuff
  backgroundcolor=\color{white},     % choose the background color
  basicstyle=\footnotesize\ttfamily, % set size + style of font
  %
  commentstyle=\color{mygray},% comment style
  identifierstyle=\color{blue},
  keywordstyle=\color{mygreen},      % keyword style
  stringstyle=\color{orange},        % string literal style
  %
  % Whitespace handling
  breakatwhitespace=false,           %
  breaklines=true,                   % sets automatic line breaking
  keepspaces=true,                   % needs (columns=flexible)?
  showspaces=false,                  % Don't add weird underscores
  showstringspaces=false,            % Same
  showtabs=false,                    % Same
  tabsize=2,	                       % 1 tab = 2 spaces
  xleftmargin=2em,                   % Properly indent listing
  %
  % Caption
  belowcaptionskip=1\baselineskip,   % Spacing to caption
  captionpos=b,                      % caption at bottom
  title=\lstname,                    % show fname w/ \lstinputlisting
  %
  % Framing & Numbering
  frame=L,	                        % adds a frame on the left
  numbers=left,                      % linum loc. (none, left, right)
  numbersep=10pt,                    % distance of linum to code
  numberstyle=\tiny\color{numgray},  % style used for the line-numbers
  rulecolor=\color{black},           %
  stepnumber=1,                      % number every line
  %
  % Misc
  language=[LaTeX]Tex,                   % default language
  % texcsstyle=*\bf\color{blue},
  % otherkeywords={$, \{, \}, \[, \]},
}

\begin{document}
\section{Writing Style Comments}
% In general: give your reader what they need, and exactly what they need. Being
% terse does not inherently mean omitting details, it really means highlighting
% the important ones
\begin{problem}[A1]
  When defining something, put it on the left side of the equals sign. Do not
  write something like
  \begin{leftbar}
    % Let ($X, \ms T$) be a topological space, and let $A \subset X$. Then define
    % the \emph{interior} of $A$ (denoted $A^\circ$ or $\mrm{int}(A)$) by
    Let
    \[
      \bigcup_{\mathclap{\substack{U \in \ms T\\ U \subset A}}} U = A^\circ
    \]
  \end{leftbar}
  to define $A^\circ$. Instead, say
  \begin{leftbar}
    Let
    \[
      A^\circ = \bigcup_{\mathclap{\substack{U \in \ms T\\ U \subset A}}} U
    \]
  \end{leftbar}
\end{problem}
\begin{problem}[A2]
  While Prof.\ Su's book does define a \emph{neighborhood} of a point $x$ to be
  some open set containing $x$, in an extended proof it's probably best to avoid
  this term. Reasons:
  \begin{enumerate}
    \item If you're using the same neighborhood repeatedly throughout your
      proof, it's much shorter to make the declaration
      \begin{leftbar}
        Let $U \in \ms T$ such that $x \in U$
      \end{leftbar}
      and use $U$ for the rest of the proof instead of saying ``the
      neighborhood'' repeatedly.
    \item It's easy to forget to specify what the neighborhood is \emph{of}.
      Remember, we've defined a neighborhood \emph{of} $x$ to be an open set
      containing $x$. Without stating what $x$ we're looking at, a neighborhood
      has no meaning. I saw a few situations on the homework where people said
      things along the lines of
      \begin{leftbar}
        ``[\ldots] so for each $x$, there exists $q$ such that [\ldots]. Then
        taking any neighborhood we see that [\ldots]''
      \end{leftbar}
      it's not clear here whether we're taking a neighborhood of $x$ or of $q$.
      Don't leave this for the reader to figure out!
    \item Finally, the definition of a neighborhood isn't universally
      standardized. Some mathematicians draw a distinction between a
      \emph{neighborhood} and an \emph{open neighborhood}:\\
      \begin{definition}
        Let $(X,\ms T)$ be a topological space. Let $x \in X$, and $N \subset
        X$. Then $N$ is called a \emph{neighborhood} of $x$ iff there exists $U
        \in \ms T$ such that $x \in U \subset N$. If $N \in \ms T$, we call $N$
        an \emph{open neighborhood} of $x$.
      \end{definition}~

      Thus, saying ``neighborhood'' has the potential to confuse your reader.
  \end{enumerate}
  For these reasons, it's probably best to use ``neighborhood'' sparingly. As a
  general rule of thumb, any time you'll need to use it more than once, you
  should define some $U \in \ms T$ instead.
\end{problem}
\begin{problem}[A3]
  Don't say ``the set $A$'' if it's already implicitly clear that $A$ is a set.
  Similarly with things like ``the point $p$.''
\end{problem}
\begin{problem}[A4]
  If you define your variables carefully at the start of a proof, you can cut
  out a lot of redundant phrases from the end of your proof. For instance, if
  you're trying to show a claim holds for all open sets $U$, then you need only
  start your proof with
  \begin{leftbar}
    Let $U \in \ms T$ be arbitrary. Then [\ldots]
  \end{leftbar}
  and conclude with
  \begin{leftbar}
    Since $U$ was arbitrarily chosen, [\ldots].
  \end{leftbar}
  When you have more than two or three variables at play, this makes a big
  difference:
  \begin{leftbar}
    Since $U,V$, and $p$ were arbitrarily chosen, [\ldots]
  \end{leftbar}
\end{problem}
\begin{problem}[A5]
  Try not to implicitly employ contradiction, unless the claim is \emph{very}
  simple. I saw a lot of things like the following:
  \begin{leftbar}
    ``We have $A$. We must have $B$, or else $\neg C$, but $A$'' (where the
    reader is left to infer that $A \implies C$).
  \end{leftbar}
  In general, the only time this is acceptable is when $C$ is a premise we have
  already established. That is, things like
  \begin{leftbar}
    ``We have $A$. Then [\ldots], and so $C$. Note that $\neg B \implies \neg
    C$, hence $B$''
  \end{leftbar}
  can sometimes be acceptable. Here're the general guidelines:
  \begin{itemize}
    \item If the proof of $A\implies C$ is $\approx 1$ step, then use the format
      above.
    \item If the proof of $A \implies C$ is $\approx 1$-$2$ steps, you can put
      the proof of $A \implies C$ in a footnote or a parenthetical. If it's on
      the longer side, definitely use a footnote.
    \item If the proof of $A \implies C$ is $\approx 3+$ steps, you should
      probably do a full proof by contradiction.
  \end{itemize}
\end{problem}
\begin{problem}[A6]
  If you're claiming something is a topology, don't refer to it as ``the
  topology'' until you've finished proving that it satisfies all the topological
  axioms. Similarly with claiming something is open, closed, etc.
\end{problem}
\begin{problem}[A8]
  In an iff proof (or analogous), you should usually define any global variables
  before you begin each arm of the proof.
\end{problem}
\begin{problem}[A9]
  For the purposes of our psets, if you find yourself saying ``in other words,''
  it's often worth checking your previous sentence to see if there's a better
  way you could phrase things that'd eliminate this redundancy.
\end{problem}
\section{Correctness}
\subsection{Exercise 3.30}
\begin{problem}[B1]
  Make sure to treat the special case where $p \in \set{x_i}_{i \in \NN}$. For
  instance, if you say the following
  \begin{leftbar}
    Let $\set{x_i}_{i\in\NN} \subset A$, and suppose $x_i \to p$. Let $U \in \ms
    T$ such that $p \in U$. Then by definition of convergence, $\exists N \in
    \NN$ such that for all $i > N$, $x_i \in U$. Thus, {\color{red}
      $\set{x_i}_{i>N} \subset (U - \set{p}) \cap A$}, hence [\ldots].
  \end{leftbar}
  this doesn't hold if for all $i > N$, $x_i = p$.
\end{problem}
\subsection{Exercise 4.3}
\begin{problem}[B2]
  When proving that $\ms B$ generates a topology $\ms T$, in verifying the
  topological axioms, be sure you show that that $\ms T$ is closed under finite
  intersection, not that $\ms B$ is (although this is also true). To do so, you
  should define arbitrary $U,V \in \ms T$, and show that $U \cap V \in \ms T$.
  Be sure not to accidentally pick $U,V \in \ms B$!
\end{problem}
\section{Comments on TeX / Notation}
\begin{problem}[C1]
  Usually, it's easier to read something like
  \[
    A^\circ = \bigcup_{\substack{U \in \mathscr T \\ U \subset A}} U
  \]
  over
  \[
    A^\circ = \bigcup_{U \in \ms T, U \subset A} U
  \]
  To do multi-line subscripts for things like \verb|\bigcup| in \LaTeX, you'll
  want to use \verb|\substack| (requires the \texttt{amsmath} package):
  \begin{lstlisting}
\[
  A^\circ = \bigcup_{\substack{U \in \ms T \\ U \subset A}} U
\]\end{lstlisting}\vspace{-3em}
  If the subscript is still really long even when using \verb|\substack| and you
  don't like the whitespace you're getting, e.g.
  \[
    A = \bigcup_{\substack{\rm look\\ \rm how\\\rm ultra-super-wide\\\rm
        this\\\rm subscript\\\rm is}} U
  \]
  you can close the gap by enclosing your subscript with \verb|\mathclap|:
  \[
    A = \bigcup_{\mathclap{\substack{\rm look\\ \rm how\\\rm ultra-super-wide\\\rm
        this\\\rm subscript\\\rm is}}} U
  \]
  or, applied to our earlier example,
  \[
    A^\circ = \bigcup_{\mathclap{U \in \mathscr T, U \subset A}} U.
  \]
  this requires the \texttt{mathtools} package, and the code is as follows:
  \begin{lstlisting}
\[
  A^\circ = \bigcup_{\mathclap{U \in \mathscr T, U \subset A}} U.
\]\end{lstlisting}\vspace{-3em}
  of course, you can put a \verb|\substack| inside a \verb|\mathclap| to combine
  the effects.
\end{problem}
\end{document}