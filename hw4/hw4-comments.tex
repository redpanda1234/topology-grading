\documentclass{fkpset}

\name{Forest Kobayashi}
\class{Topology}
\duedate{03/03/2019}
\assignment{HW 4 Comments}

\chead{HW 4 Comments}
\rhead{Math 147 -- Spring 2019}

\usepackage{hyperref}
\usepackage{fkmisc}

\begin{document}
Announcements:
\begin{itemize}
  \item Send errata to me
  \item When2Meet for writing style tutoring stuff
\end{itemize}

\section{Writing Style Comments}
\begin{problem}[A1]
  TL;DR --- Don't be afraid to introduce new notation if you feel it's helpful.
  But if you choose to do so, you should \emph{define it explicitly} at the
  beginning of your proof.\\

  Here's an example. In a problem like 4.31, you might consider using angled
  brackets to distinguish points from intervals. E.g., you could use $\ip{a,b}$
  to denote a point in $\RR^2$, and $\pn{a,b}$ to denote the interval $\pn{a,b}
  \subset \RR$ (In \LaTeX, $\ip{a,b}$ can be generated with
  \verb|$\langle a,b \rangle$|).\\

  If you choose this approach, you should begin your proof with something like
  ``to avoid confusion, we will denote points of $\RR^2$ with angled brackets
  (e.g.\ $\ip{a,b}$), and intervals in $\RR$ with the usual $(a,b)$, $[a,b]$,
  etc.'' \\

  If you choose not to create new notation, you should be sure to explicitly
  draw attention to when you're treating $(a,b)$ as a point vs.\ as an interval.
  There are a few things you can do to make this natural:\\[-.75em]
  \begin{enumerate}[label=(\arabic*)]
    \item When you declare $(a,b)$, you should make the domain explicit by
      saying ``let $(a,b) \in \RR^2$'' or ``let $(a,b) \subset \RR$.'' In fact,
      you should \emph{always} do this with your variables, it's just
      particularly relevant here.
    \item Use variable symbols consistently. If you previously used $(a,b)$ to
      refer to a point in $\RR^2$, you should not overwrite that declaration
      later by saying ``let $(a,b) \subset \RR$.'' Instead, you should choose
      different symbols. Usually $(a,b)$ is used for intervals, and $(x,y)$ for
      points --- consider employing this scheme.
    \item Consider using vector notation. E.g.,
      \begin{leftbar}
        Let $(X_{\rm sq}, \ms T_{\rm sq})$ denote the lexicographically ordered
        square, and let $\ms B_{\rm sq}$ be the canonical basis for $\ms T_{\rm
          sq}$. Denote elements of $\ms B_{\rm sq}$ by $(\mb u, \mb v)$, where
        $\mb u, \mb v \in X_{\rm sq}$ (using square brackets as appropriate if
        $\mb u = \mb 0$ or $\mb v = \mb 1$).
      \end{leftbar}
  \end{enumerate}
\end{problem}
\begin{problem}[A2]
  Elegance, and combining similar casework
\end{problem}
\begin{problem}[A3]
  ``Consider'' and ``take''
\end{problem}
\begin{problem}[A4]
  How to identify when to give a WTS. Also, justification comes before
  assertions. Assertions $\neq$ WTSs.
\end{problem}
\section{Correctness}
\subsection{Exercise 4.41}
\begin{problem}[B2]
  Don't label $\RR_\alpha$! This is only for when the sets in our product are
  distinct
\end{problem}
\subsection{Exercise 5.5}
\begin{problem}[B1]
  It's a bit hazardous to try and make an argument based on the \emph{form} of
  arbitrary open sets in $\RR_{\rm LL}$.
\end{problem}
\section{Comments on TeX / Notation}
\end{document}