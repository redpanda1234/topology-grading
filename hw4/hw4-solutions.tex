\documentclass{fkpset}

\newgeometry{bmargin=1in, tmargin=1.25in, lmargin=.75in, rmargin=.75in}
\fancyhfoffset[R]{.05cm}

\name{Forest Kobayashi}
\class{Math 147}
\duedate{03/03/2019}
\assignment{HW 4 Solutions}

\chead{HW 4 Solutions}
\rhead{Math 147 -- Spring, 2019}

\lfoot{Sunday, March 3rd 2019}

\renewcommand{\thesubfigure}{\roman{subfigure}}

\newcommand{\sq}{{\rm sq}}
\newcommand{\RRLL}{\ensuremath \RR_{\rm LL}}
\newcommand{\tprod}{\ensuremath \ms T_{\rm prod}}
\newcommand{\bprod}{\ensuremath \ms B_{\rm prod}}
\newcommand{\tbox}{\ensuremath \ms T_{\rm box}}
\newcommand{\bbox}{\ensuremath \ms B_{\rm box}}
\newcommand{\tstar}{\ensuremath \ms T_{\star}}
\newcommand{\bstar}{\ensuremath \ms B_{\star}}

\renewcommand{\qed}{\hspace{.85\textwidth} $\blacksquare$}

\newenvironment{why}{\begin{adjustwidth}{0.01\linewidth}{0.05\linewidth}~}%
  {\end{adjustwidth}}

\begin{document}
\pagestyle{plain}
\pagestyle{fancy}
  \vspace{-3cm}
  \begin{table}[H]
    \centering
    \begin{tabular}{@{}lcccccr@{}}\toprule
      Problems & 4.31(F) & 4.41 & 5.1 & 5.5 & 5.9 & Total \\ \midrule
      Points   &         &      &     &     &     &       \\ \bottomrule
    \end{tabular}
  \end{table}
  \vspace{1cm}

% --------------------------- Problem 1 ---------------------------- %
  \begin{problem}[4.31(F)]
    Consider the following subspace of the lexicographically ordered
    square:
    \[
      F = \set{(x,1) \MID 0 < x < 1}.
    \]
    As a set, it is a line. Describe its relative topology, noting any
    connections to the topologies you have seen already.
  \end{problem}
  % There was a slight spacing difference for my leftbar environment
  % that doesn't really merit changing in generality, so I just went
  % with this hack.
  \begin{why}
    \small
    \begin{leftbar}
      \textbf{General Note:} Honestly, the edge cases here make this
      problem a bit of a mess. Feel free to gloss over them by drawing
      diagrams for each case instead of actually writing things out
      formally. I've included everything here for the sake of
      completeness, but you should feel free to omit some parts. Just
      make it clear to me that you know \emph{why} everything has the
      form it should! \texttt{:)}
    \end{leftbar}
    \begin{leftbar}
      \textbf{Notational Notes:} For the purposes of this problem, we
      will use boldface/angled brackets to denote points, and
      parentheses for open intervals. E.g., $\Pp = \ip{a,b} \in
      \RR^2$, whereas $\pn{a,b} \subset \RR$. Note, under this
      notational scheme, ``open intervals'' in the lexicographically
      ordered square will thus be denoted by something like
      $\pn{\ip{a_0, b_0}, \ip{a_1, b_1}}$:
      \[
        \pn[Big]{\ip{a_0, b_0}, \ip{a_1, b_1}} =
        \set[Big]{\ip{x,y} \MID \ip{a_0, b_0} < \ip{x,y} < \ip{a_1,
            b_1}}
      \]
      using square brackets when appropriate. Note that if we have
      previously defined $\Pp_0 = \ip{a_0, b_0}$, $\Pp_1 = \ip{a_1,
        b_1}$, then we would just write this as $\pn{\Pp_0, \Pp_1}$.

      Also, injection will be denoted $f : A \into B$, surjection by
      $f : A \onto B$, and bijection by $f : A \bij B$.
    \end{leftbar}
  \end{why}
  \begin{solution}
    Let $(X_\sq, \ms T_\sq)$ be the lexicographically ordered square,
    and $\ms B_\sq$ be the canonical basis for $\ms T_\sq$. Let $\ms
    T_F$ be the relative topology on $F$ inherited from $X_\sq$. Also
    let $(I, \ms T_I)$ be the interval $(0,1)$ together with the
    subspace topology inherited from $\RRLL$. Then we claim $(F, \ms
    T_F)$ is ``equivalent'' to $(I, \ms T_I)$ (denoted $\ms T_F \cong
    \ms T_I$).\footnote{Note: Technically, the formal term to use here
      is ``homeomorphic.'' Homeomorphism is the notion of a
      topological ``equivalence,'' or in the language of Category
      Theory, ``isomorphism in the category of topological spaces''.
      This won't be covered formally until section 8.3, but the idea
      is that homeomorphisms are bijections preserving open sets. This
      property of ``preserving open sets'' will become our topological
      definition of continuity, and in this context, a homeomorphism
      will be a continuous function with a continuous inverse.}

    \emph{Proof:}
    By theorem 4.30,
    \[
      \ms B_F = \set{B \cap F \MID B \in \ms B_\sq}
    \]
    is a basis for $\ms T_F$. Thus, to show $\ms T_F \cong \ms T_I$,
    it will suffice to characterize elements of $\ms B_F$ (Claim 1)
    and put them in correspondence with a basis for $\ms T_I$ (Claim
    2).

    \emph{Claim 1:} Elements of $\ms B_F$ are of the following
    forms:\footnote{Note, (2) and (3) can also be written as $\pn{a,b}
      \times \set{1}$ and $\bp{a,b} \times \set{1}$,
      respectively}
    \begin{enumerate}[label=(\arabic*)]
      \item $\varnothing$
      \item $\set{\ip{x,1} \in F \MID a < x < b}\quad$ (where $a,b \in
        (0,1)$)
      \item $\set{\ip{x,1} \in F \MID a \leq x < b}\quad$ (where $a,b
        \in (0,1)$)
    \end{enumerate}
    \emph{Proof of Claim 1:} Let $B_F \in \ms B_F$ be arbitrary. Then
    $\exists B_\sq \in \ms B_\sq$ such that $B_F = B_\sq \cap F$. Let
    $\mb 0 = \ip{0,0}$ and $\mb 1 = \ip{1,1}$. By definition of $\ms
    B_\sq$, $B_\sq$ is of one of the following forms:\clearpage
    \begin{enumerate}
      \item $\bp[Big]{\mb 0, \ip{a_1, b_1}}$,
      \item $\pn[Big]{\ip{a_0, b_0}, \ip{a_1, b_1}}$,
      \item $\pb[Big]{\ip{a_0, b_0}, \mb 1}$
    \end{enumerate}
    We proceed by casework.
    \begin{enumerate}
      \item Suppose $B_\sq = \bp[Big]{\mb 0, \ip{a_1, b_1}}$. Then
        \[
          B_\sq \cap F = \set[Big]{\ip{x,1} \in X_\sq \MID \mb 0 <
            \ip{x,1} < \ip{a_1, b_1}}.
        \]
        We will consider the cases of $\ip{a_1,b_1} \leq \ip{0,1}$ and
        $\ip{0,1} < \ip{a_1,b_1}$. Note that these cases are disjoint
        and exhaustive.
        \begin{enumerate}[label=\roman*)]
          \item If $\ip{a_1,b_1} \leq \ip{0,1}$, $F \cap B =
            \varnothing$ (see diagram). This is of form
            (1).\footnote{Note this holds even if $\ip{a_1,b_1} =
              \ip{0,1}$, because the green endpoint is non-inclusive.}
            \begin{figure}[H]
              \centering
              \begin{tikzpicture}[scale=4]
                % Coordinates for F and X
                \path
                coordinate (00) at (0,0)
                coordinate (01) at (0,1)
                coordinate (10) at (1,0)
                coordinate (11) at (1,1);

                \draw[green, thick] (01) -- (11);
                \draw[thick, red] (00) -- (0,.6);

                \fill[gray!10!white] (00) -- (01) -- (11) -- (10) -- (00) -- cycle;
                \draw[draw=red, fill=red] (00) circle (.5pt) node[left, yshift=-4pt] {$\color{red} \mb 0$};
                \draw[draw=green, fill=white] (01) circle (.5pt) node[left, yshift=4pt] {$\color{green}\ip{0,1}$};
                \draw[gray!70!white, fill=gray!20!white] (10) circle (.5pt) node[right, yshift=-4pt] {$\color{gray!70!white} \ip{1,0}$};
                \draw[draw=green, fill=white] (11) circle (.5pt) node[right, yshift=4pt] {$\color{green}\mb 1$};

                \draw[red, fill=white] (0,.6) circle (.5pt) node[left, xshift=-6pt] {$\color{red} \ip{a_1, b_1}$};


                \draw [
                  red,
                  postaction=decorate,
                    decoration={
                      markings,
                      mark=at position .5 with {\arrow{angle 45}}
                    }
                ] (00) -- (0,.6);

                \draw [
                  red,
                  decorate,
                  decoration={
                    brace,
                    amplitude=6pt
                  },
                  xshift=-1pt,
                  yshift=0pt
                ] (0,.025) -- (0,.575) node [red, midway, xshift=-0.4cm] {\footnotesize $B$};

                \draw [
                  green,
                  decorate,
                  decoration={
                    brace,
                    amplitude=6pt
                  },
                  xshift=0pt,
                  yshift=1pt
                ] (0.025,1) -- (.975,1) node [green, midway, yshift=0.4cm] {\footnotesize $F$};
              \end{tikzpicture}
              \caption{${\color{green}F} \cap {\color{red}B} = \color{blue}
                \varnothing$.}
            \end{figure}
          \item Suppose $\ip{0,1} < \ip{a_1,b_1}$. Then
            ${\color{green}F} \cap {\color{red}B} =
            {\color{blue}\set{\ip{x,1} \in X_\sq \MID x \in (0,a_1)}}
            = {\color{blue}\pn{0,a_1} \times \set{1}}$ (see
            diagram).\footnote{Note that $\ip{0,1} \not\in F \cap B$
              by definition of $F$.} This is of form (2).
            \begin{figure}[H]
              \centering
              \begin{tikzpicture}[scale=4]
                % Coordinates for F and X
                \path
                coordinate (00) at (0,0)
                coordinate (01) at (0,1)
                coordinate (10) at (1,0)
                coordinate (11) at (1,1)
                coordinate (a1) at (.3,0)
                coordinate (rmax) at (.3,.7)
                coordinate (bmax) at (.3,1);

                \fill[gray!10!white] (a1) -- (bmax) -- (11) -- (10) -- (a1) -- cycle;

                \fill[red!15!white] (00) -- (01) -- (bmax) -- (a1) -- (00) -- cycle;

                \draw[green, thick] (01) -- (11);
                \draw[red] (01) -- (00) -- (a1);
                \draw[blue, thick] (01) -- (bmax);

                \draw[red, dashed] (rmax) -- (bmax);

                \draw[
                  red,
                  postaction={
                    decorate,
                    decoration={
                      markings,
                      mark=at position .5 with {\arrow{angle 45}}
                    }
                  }
                ] (00)--(01);

                \draw[
                  red,
                  postaction={
                    decorate,
                    decoration={
                      markings,
                      mark=at position .5 with {\arrow{angle 45}}
                    }
                  }
                ] (a1) -- (rmax);

                \draw[red, fill=white] (rmax) circle (.5pt) node[right, xshift=4pt] {$\color{red} \ip{a_1, b_1}$};

                \draw [
                  blue,
                  decorate,
                  decoration={
                    brace,
                    amplitude=6pt
                  },
                  xshift=0pt,
                  yshift=.5pt
                ] (.025,1) -- (0.275,1) node [blue, midway, yshift=0.4cm] {\footnotesize $F \cap B$};

                \draw[draw=red, fill=red] (00) circle (.5pt) node[left, yshift=-4pt] {$\color{red} \mb 0$};

                \draw[draw=blue, fill=white] (01) circle (.5pt) node[above left, yshift=2pt] {$\color{blue}\ip{0,1}$};
                \draw[draw=blue, fill=white] (bmax) circle (.5pt) node[above right, yshift=2pt] {$\color{blue}\ip{a_1,1}$};
                \draw[gray!70!white, fill=gray!20!white] (10) circle (.5pt) node[right, yshift=-4pt] {$\color{gray!70!white} \ip{1,0}$};
                \draw[draw=green, fill=white] (11) circle (.5pt) node[right, yshift=4pt] {$\color{green}\mb 1$};

              \end{tikzpicture}
              \caption{${\color{green}F} \cap {\color{red}B} =
                {\color{blue} \pn{0, a_1} \times \set{1}}$.}
            \end{figure}
        \end{enumerate}
        hence, if $B_\sq$ is of form (a), then $B_\sq \cap F$ is of
        the desired form. \cmark
      \item Now, suppose $B_\sq = \pn{\ip{a_0, b_0}, \ip{a_1, b_1}}$.
        Omitting the details, we have the following cases: (i) $a_0 =
        a_1$, (ii) $a_0 \neq a_1$ and $b_0 = 1$, and (iii) $a_0 \neq
        a_1$ and $b_0 \neq 1$.
        \begin{figure}[H]
          \centering
          \begin{subfigure}{.32\linewidth}
            \centering
            \begin{tikzpicture}[scale=3.5]
              % Coordinates for F and X
              \path
              coordinate (00) at (0,0)
              coordinate (01) at (0,1)
              coordinate (10) at (1,0)
              coordinate (11) at (1,1)
              coordinate (a1) at (.3,0)
              coordinate (rmin) at (.3,.2)
              coordinate (rmax) at (.3,.8)
              coordinate (bmax) at (.3,1);

              \fill[gray!10!white] (00) -- (01) -- (11) -- (10) -- (00) -- cycle;

              \draw[green, thick] (01) -- (11);
              \draw[red, dashed] (a1) -- (rmin) (rmax) -- (bmax);

              \draw[
              red,
              postaction={
                decorate,
                decoration={
                  markings,
                  mark=at position .5 with {\arrow{angle 45}}
                }
              }
              ] (rmin)--(rmax);

              \draw[red, fill=white] (rmin) circle (.5pt) node[right, xshift=4pt] {$\color{red} \ip{a_0, b_0}$};

              \draw[red, fill=white] (rmax) circle (.5pt) node[right, xshift=4pt] {$\color{red} \ip{a_1, b_1}$};

              \draw[draw=gray!70!white, fill=gray!20!white] (00) circle (.5pt) node[left, yshift=-4pt] {$\color{gray!70!white} \mb 0$};

              \draw[draw=green, fill=white] (01) circle (.5pt) node[left, yshift=4pt] {$\color{green}\ip{0,1}$};
              \draw[gray!70!white, fill=gray!20!white] (10) circle (.5pt) node[right, yshift=-4pt] {$\color{gray!70!white} \ip{1,0}$};
              \draw[draw=green, fill=white] (11) circle (.5pt) node[right, yshift=4pt] {$\color{green}\mb 1$};

            \end{tikzpicture}
            \caption{\texttt{if} $a_0 = a_1$, ${\color{green}F} \cap
              {\color{red}B} = {\color{blue} \varnothing}$.}
          \end{subfigure}
          \begin{subfigure}{.32\linewidth}
            \centering
            \begin{tikzpicture}[scale=3.5]
              % Coordinates for F and X
              \path
              coordinate (00) at (0,0)
              coordinate (01) at (0,1)
              coordinate (10) at (1,0)
              coordinate (11) at (1,1)
              coordinate (a0) at (.3,0)
              coordinate (a1) at (.61,0)
              coordinate (rmin) at (.3,1)
              coordinate (rmax) at (.61,.6)
              coordinate (bmin) at (.3,1)
              coordinate (bmax) at (.61,1);

              \fill[gray!10!white] (00) -- (01) -- (11) -- (10) -- (00) -- cycle;
              \fill[red!15!white] (a0) -- (bmin) -- (bmax) -- (a1) -- (a0) -- cycle;

              \draw[green, thick] (01) -- (11);
              \draw[red, dashed] (a0) -- (rmin) (rmax) -- (bmax);
              \draw[blue, thick] (bmin) -- (bmax);

              \draw[
              red,
              postaction={
                decorate,
                decoration={
                  markings,
                  mark=at position .5 with {\arrow{angle 45}}
                }
              }
              ] (a1)--(rmax);

              \draw[red, fill=white] (rmin) circle (.5pt) node[below left, xshift=-1pt] {\scriptsize$\color{red} \ip{a_0, b_0}$};

              \draw[red, fill=white] (rmax) circle (.5pt) node[right, xshift=2pt] {\scriptsize$\color{red} \ip{a_1, b_1}$};

              \draw[blue, fill=white] (bmin) circle (.5pt) node[above, yshift=2pt] {\scriptsize $\color{blue} \ip{a_0, 1}$};

              \draw[blue, fill=white] (bmax) circle (.5pt) node[above, yshift=2pt] {\scriptsize $\color{blue} \ip{a_1, 1}$};

              \draw[draw=gray!70!white, fill=gray!20!white] (00) circle (.5pt) node[left, yshift=-4pt] {$\color{gray!70!white} \mb 0$};

              \draw[draw=green, fill=white] (01) circle (.5pt) node[left, yshift=4pt] {$\color{green}\ip{0,1}$};
              \draw[gray!70!white, fill=gray!20!white] (10) circle (.5pt) node[right, yshift=-4pt] {$\color{gray!70!white} \ip{1,0}$};
              \draw[draw=green, fill=white] (11) circle (.5pt) node[right, yshift=4pt] {$\color{green}\mb 1$};

            \end{tikzpicture}
            \caption{\texttt{elif} $b_0=1$, ${\color{green} F} \cap {\color{red}
                B} = {\color{blue} \pn{a_0, a_1} \times \set{1}}$}
          \end{subfigure}
          \begin{subfigure}{.32\linewidth}
            \centering
            \begin{tikzpicture}[scale=3.5]
              % Coordinates for F and X
              \path
              coordinate (00) at (0,0)
              coordinate (01) at (0,1)
              coordinate (10) at (1,0)
              coordinate (11) at (1,1)
              coordinate (a0) at (.3,0)
              coordinate (a1) at (.61,0)
              coordinate (rmin) at (.3,.6)
              coordinate (rmax) at (.61,.4)
              coordinate (bmin) at (.3,1)
              coordinate (bmax) at (.61,1);

              \fill[gray!10!white] (00) -- (01) -- (11) -- (10) -- (00) -- cycle;
              \fill[red!15!white] (a0) -- (bmin) -- (bmax) -- (a1) -- (a0) -- cycle;

              \draw[green, thick] (01) -- (11);
              \draw[red, dashed] (a0) -- (rmin) (rmax) -- (bmax);
              \draw[blue, thick] (bmin) -- (bmax);

              \draw[
              red,
              postaction={
                decorate,
                decoration={
                  markings,
                  mark=at position .5 with {\arrow{angle 45}}
                }
              }
              ] (rmin)--(bmin);

              \draw[
              red,
              postaction={
                decorate,
                decoration={
                  markings,
                  mark=at position .5 with {\arrow{angle 45}}
                }
              }
              ] (a1)--(rmax);

              \draw[red, fill=white] (rmin) circle (.5pt) node[left, xshift=-1pt] {\scriptsize$\color{red} \ip{a_0, b_0}$};

              \draw[red, fill=white] (rmax) circle (.5pt) node[right, xshift=2pt] {\scriptsize$\color{red} \ip{a_1, b_1}$};

              \draw[blue, fill=blue] (bmin) circle (.5pt) node[above, yshift=2pt] {\scriptsize $\color{blue} \ip{a_0, 1}$};

              \draw[blue, fill=white] (bmax) circle (.5pt) node[above, yshift=2pt] {\scriptsize $\color{blue} \ip{a_1, 1}$};

              \draw[draw=gray!70!white, fill=gray!20!white] (00) circle (.5pt) node[left, yshift=-4pt] {$\color{gray!70!white} \mb 0$};

              \draw[draw=green, fill=white] (01) circle (.5pt) node[left, yshift=4pt] {$\color{green}\ip{0,1}$};
              \draw[gray!70!white, fill=gray!20!white] (10) circle (.5pt) node[right, yshift=-4pt] {$\color{gray!70!white} \ip{1,0}$};
              \draw[draw=green, fill=white] (11) circle (.5pt) node[right, yshift=4pt] {$\color{green}\mb 1$};

            \end{tikzpicture}
            \caption{\texttt{else}, ${\color{green} F} \cap {\color{red} B} =
              {\color{blue} \bp{a_0, a_1} \times \set{1}}$}
          \end{subfigure}
          \caption{Subcases}
        \end{figure}
        these have the forms (1), (2), and (3), respectively.
      \item Finally, suppose $B_\sq$ has the form $\pb[Big]{\ip{a_0,
            b_0}, \mb 1}$. Then we have the following cases: (i) $a_0
        = 1$, (ii) $a_0 < 1$ and $b_0=1$, and (iii) $a_0 < 1$ and $b_0
        < 1$. Note that these cases are disjoint and exhaustive.
        \begin{figure}[H]
          \centering
          \begin{subfigure}{.32\linewidth}
            \centering
            \begin{tikzpicture}[scale=3.5]
              % Coordinates for F and X
              \path
              coordinate (00) at (0,0)
              coordinate (01) at (0,1)
              coordinate (10) at (1,0)
              coordinate (11) at (1,1)
              coordinate (a1) at (1,0)
              coordinate (rmin) at (1,.2)
              coordinate (rmax) at (1, 1)
              coordinate (bmax) at (1, 1);

              \fill[gray!10!white] (00) -- (01) -- (11) -- (10) -- (00) -- cycle;

              \draw[green, thick] (01) -- (11);
              \draw[red, dashed] (a1) -- (rmin) (rmax) -- (bmax);

              \draw[
              red,
              postaction={
                decorate,
                decoration={
                  markings,
                  mark=at position .5 with {\arrow{angle 45}}
                }
              }
              ] (rmin)--(rmax);

              \draw[red, fill=white] (rmin) circle (.5pt) node[right, xshift=4pt] {$\color{red} \ip{a_0, b_0}$};

              \draw[draw=gray!70!white, fill=gray!20!white] (00) circle (.5pt) node[left, yshift=-4pt] {$\color{gray!70!white} \mb 0$};

              \draw[draw=green, fill=white] (01) circle (.5pt) node[left, yshift=4pt] {$\color{green}\ip{0,1}$};
              \draw[gray!70!white, fill=gray!20!white] (10) circle (.5pt) node[right, yshift=-4pt] {$\color{gray!70!white} \ip{1,0}$};
              \draw[draw=green, fill=white] (11) circle (.5pt) node[right, yshift=4pt] {${\color{green}\mb 1}$};

            \end{tikzpicture}
            \caption{\texttt{if} $a_0 = 1$, ${\color{green}F} \cap
              {\color{red}B} = {\color{blue} \varnothing}$.}
          \end{subfigure}
          \begin{subfigure}{.32\linewidth}
            \centering
            \begin{tikzpicture}[scale=3.5]
              % Coordinates for F and X
              \path
              coordinate (00) at (0,0)
              coordinate (01) at (0,1)
              coordinate (10) at (1,0)
              coordinate (11) at (1,1)
              coordinate (a0) at (.53,0)
              coordinate (a1) at (1,0)
              coordinate (rmin) at (.53,1)
              coordinate (rmax) at (1, 1)
              coordinate (bmin) at (.53,1)
              coordinate (bmax) at (1,1);

              \fill[gray!10!white] (00) -- (01) -- (11) -- (10) -- (00) -- cycle;
              \fill[red!15!white] (a0) -- (bmin) -- (bmax) -- (a1) -- (a0) -- cycle;

              \draw[green, thick] (01) -- (11);
              \draw[red, dashed] (a0) -- (rmin) (rmax) -- (bmax);
              \draw[red] (rmin) -- (bmin);
              \draw[blue, thick] (bmin) -- (bmax);

              \draw[
              red,
              postaction={
                decorate,
                decoration={
                  markings,
                  mark=at position .5 with {\arrow{angle 45}}
                }
              }
              ] (a1)--(rmax);

              \draw[red, fill=white] (rmax) circle (.5pt) node[right, xshift=2pt] {};

              \draw[blue, fill=white] (bmin) circle (.5pt) node[above, yshift=2pt] {\scriptsize ${\color{blue} \ip{a_0, 1}} = {\color{red} \ip{a_0, b_0}}$};

              \draw[blue, fill=white] (bmax) circle (.5pt) node[above, yshift=2pt] {};

              \draw[draw=gray!70!white, fill=gray!20!white] (00) circle (.5pt) node[left, yshift=-4pt] {$\color{gray!70!white} \mb 0$};

              \draw[draw=green, fill=white] (01) circle (.5pt) node[left, yshift=4pt] {$\color{green}\ip{0,1}$};
              \draw[red,fill=red] (10) circle (.5pt) node[right, yshift=-4pt] {$\color{red} \ip{1,0}$};
              \draw[draw=green, fill=white] (11) circle (.5pt) node[right, yshift=4pt] {${\color{green}\mb 1}$};

            \end{tikzpicture}
            \caption{\texttt{elif} $b_0 = 1$, ${\color{green} F} \cap
              {\color{red} B} = {\color{blue} \pn{a_0, 1} \times \set{1}}$}
          \end{subfigure}
          \begin{subfigure}{.32\linewidth}
            \centering
            \begin{tikzpicture}[scale=3.5]
              % Coordinates for F and X
              \path
              coordinate (00) at (0,0)
              coordinate (01) at (0,1)
              coordinate (10) at (1,0)
              coordinate (11) at (1,1)
              coordinate (a0) at (.53,0)
              coordinate (a1) at (1,0)
              coordinate (rmin) at (.53,.6)
              coordinate (rmax) at (1, 1)
              coordinate (bmin) at (.53,1)
              coordinate (bmax) at (1,1);

              \fill[gray!10!white] (00) -- (01) -- (11) -- (10) -- (00) -- cycle;
              \fill[red!15!white] (a0) -- (bmin) -- (bmax) -- (a1) -- (a0) -- cycle;

              \draw[green, thick] (01) -- (11);
              \draw[red, dashed] (a0) -- (rmin) (rmax) -- (bmax);
              \draw[red] (rmin) -- (bmin);
              \draw[blue, thick] (bmin) -- (bmax);

              \draw[
              red,
              postaction={
                decorate,
                decoration={
                  markings,
                  mark=at position .5 with {\arrow{angle 45}}
                }
              }
              ] (rmin)--(bmin);

              \draw[
              red,
              postaction={
                decorate,
                decoration={
                  markings,
                  mark=at position .5 with {\arrow{angle 45}}
                }
              }
              ] (a1)--(rmax);

              \draw[red, fill=white] (rmin) circle (.5pt) node[below left, xshift=-1pt] {\scriptsize$\color{red} \ip{a_0, b_0}$};

              \draw[red, fill=white] (rmax) circle (.5pt) node[right, xshift=2pt] {};

              \draw[blue, fill=blue] (bmin) circle (.5pt) node[above, yshift=2pt] {\scriptsize $\color{blue} \ip{a_0, 1}$};

              \draw[blue, fill=white] (bmax) circle (.5pt) node[above, yshift=2pt] {};

              \draw[draw=gray!70!white, fill=gray!20!white] (00) circle (.5pt) node[left, yshift=-4pt] {$\color{gray!70!white} \mb 0$};

              \draw[draw=green, fill=white] (01) circle (.5pt) node[left, yshift=4pt] {$\color{green}\ip{0,1}$};
              \draw[red,fill=red] (10) circle (.5pt) node[right, yshift=-4pt] {$\color{red} \ip{1,0}$};
              \draw[draw=green, fill=white] (11) circle (.5pt) node[right, yshift=4pt] {\scriptsize ${\color{green}\mb 1} = {\color{red} \ip{a_1, b_1}} = {\color{blue} \ip{a_1, 1}}$};

            \end{tikzpicture}
            \caption{\texttt{else}, ${\color{green} F} \cap
              {\color{red} B} = {\color{blue} \bp{a_0, 1} \times
                \set{1}}$}
          \end{subfigure}
          \caption{Yet more subcases}
        \end{figure}
        which are of the forms (1), (2), and (3), respectively.
    \end{enumerate}
    Since cases (a) -- (c) are exhaustive, This proves the claim.
    (\emph{Phew\ldots finally!}

    \begin{leftbar}
      \color{red} Note: you can stop here if you're doing your
      rewrite, as long as you just make a brief note about the
      similarity of $\ms B_F$ to a basis for $\ms T_I$.
    \end{leftbar}

    \emph{Claim 2:} We can establish a natural bijection between $\ms
    B_F$ and a basis $\ms B_I$ for $\ms T_I$, and extend this to a
    function $f : F \bij I$ that respects the topologies $\ms T_F, \ms
    T_I$.

    \emph{Proof of Claim 2:} Let $\ms B_I$ be given by
    \[
      \ms B_I = \set{\varnothing} \cup \set[Big]{\pn{a,b} \subset
        \pn{0,1}} \cup \set[Big]{\bp{a,b} \subset \pn{0,1}}.
    \]
    Observe that $\ms B_I$ is indeed a basis for $\ms
    T_I$.\footnote{In particular, the rightmost term is the collection
      of standard basis sets for $\ms T_I$, and the left two terms can
      be obtained by closure under arbitrary union.} Thus, $f : \ms
    B_I \bij \ms B_F$ defined by
    \[
      f(B_I) =
      \begin{cases}
        \varnothing & \text{if } B_I = \varnothing \\
        B_I \times \set{1} & \text{otherwise }
      \end{cases}
    \]
    is a bijection with inverse given by
    \[
      \hspace{6.5cm}f^{-1}(B_F) = \pi_x(B_F) \qquad \qquad
      \text{(where $\pi_x$ is the canonical projection).}
    \]
    We want to show $\forall U_I\in \ms T_I$, $f(U_I) \in \ms T_F$,
    and $\forall V_F \in \ms T_F$, $f^{-1}(V_F) \in \ms T_I$. Let $U_I
    \in \ms T_I$ be arbitrary. Then since $\ms B_I$ is a basis, there
    exists a collection of basis sets $\set{B_{I, \alpha}}_{\alpha \in
      \lambda} \subset \ms B_I$ such that
    \[
      \bigcup_{\alpha \in \lambda} B_{I,\alpha} = U_I.
    \]
    Hence
    \begin{align*}
      f(U_I)
      &= f\pn{
        \bigcup_{\alpha \in \lambda} B_{I,\alpha} } & (\text{substituting for $U_I$})\\
      &= \bigcup_{\alpha \in \lambda} f\pn{B_{I,\alpha}} & \text{(prop.\ of $\cup$)}
    \end{align*}
    For each $\alpha \in \lambda$, $f\pn{B_{I,\alpha}} \in \ms B_F$
    (by definition of $f$). Hence, $\set{B_{F,\alpha}}_{\alpha \in
      \lambda}$ defined by
    \[
      B_{F,\alpha} = f\pn{B_{I,\alpha}} \qquad (\alpha \in \lambda)
    \]
    is a subset of $\ms B_F$. It follows that
    \[
      f(U_I) = \bigcup_{\alpha \in \lambda} B_{F,\alpha}
    \]
    is open in $(F, \ms T_F)$. The proof for arbitrary $V_F \in \ms
    T_F$ follows analogously.

    Hence, $f$ respects the topologies on $F,I$, so we have
    \[
      (F, \ms T_F) \cong (I, \ms T_I)
    \]
    as desired.
  \end{solution}
  \clearpage

% --------------------------- Problem 2 ---------------------------- %

  \begin{problem}[4.41]
    Let $\RR^\omega$ be the countable product of copies of $\RR$. So
    every point in $\RR^\omega$ is a sequence $x_1, x_2, x_3, \ldots$.
    Let $A \subset \RR^\omega$ be the set consisting of all points
    with only positive coordinates. Show that in the product topology,
    $\Oo = \pn{0,0,0,\ldots}$ is a limit point of the set $A$, and
    there is a sequence of points in $A$ converging to $\Oo$. Then
    show that in the box topology, $\Oo = \pn{0,0,0,\ldots}$ is a
    limit point of the set $A$, but there is no sequence of points in
    $A$ converging to $\Oo$.
  \end{problem}
  \begin{why}
    \begin{leftbar}
      \textbf{General Note:} In this problem, we pick an arbitrary $U
      \in \ms T$. Lots of people assumed that $U$ needs to be of the
      form
      \[
        \prod_{i \in \NN} U_i
      \]
      where for each $i$, either $U_i = (a_i, b_i)$ or $U_i = \RR$.
      This isn't how the box/product topologies were defined (we only
      know $U_i \in \ms T_{\rm std}$, not $U_i \in \ms B_{\rm std}$).
      However, I didn't take off points for this assumption, since
      this case captures the essential ideas of the proof.
    \end{leftbar}
    \begin{leftbar}
      \textbf{Notational Note:} Let $A$ be a subset of a topological
      space. Then we will use $\ms L(A)$ to denote the limit points of
      $A$.
    \end{leftbar}
  \end{why}
  \begin{solution}
    Let $\ms \tprod, \ms \tbox$ be the product and box topologies on
    $\RR^\omega$, respectively. Denote their corresponding bases by
    $\bprod, \bbox$. We first show that in either topology $\Oo \in
    \ms L(A)$, and then show the results about sequences.
    \begin{enumerate}[label=(\arabic*)]
      \item WTS $\Oo \in \ms L(A)$. The following proof works in both
        $\tprod$ and $\tbox$.\footnote{This is because $\tbox$ is
          \emph{finer} than $\tprod$, and the proof works in $\tbox$.}
        Hence, let $\tstar$ refer to either one of them, and let
        $\bstar$ be the corresponding basis.\footnote{Formally, ``let
          $\tstar \in \set{\tprod, \tbox}$ be arbitrary'' --- but this
          level of formality might distract the reader more than it
          helps, so I omit it.} We proceed by definition of a limit
        point.

        Let $U \in \tstar$ such that $\Oo \in U$. Then there exists $B
        \in \bstar$ such that $B \subset U$, and $\Oo \in B$. By
        definition of $\ms B_\star$, $B$ has the form
        \[
          B = \prod_{i \in \NN} V_i.
        \]
        where each $V_i$ is open in $\RR_{\rm std}$.\footnote{In the
          product topology, we require all but finitely many of the
          $V_i$ to be copies of $\RR$. But note, $\RR$ is open in
          $\RR_{\rm std}$, so we lose no generality in this
          statement.}
        $\Oo \in B$ implies $0 \in V_i$ for each $i$, so there exists
        $(a_i, b_i) \subset \RR$ with $0 \in (a_i,b_i) \subset \RR$.
        Hence taking
        \[
          \Aa = \pn{\frac{b_1}{2}, \frac{b_2}{2}, \ldots,
            \frac{b_i}{2}, \ldots}
        \]
        we see that $\Aa \in B$.
      \item
    \end{enumerate}
  \end{solution}
  \clearpage

% --------------------------- Problem 3 ---------------------------- %
  \begin{problem}[5.1]
    A space $(X, \ms T)$ is $T_1$ if and only if every point in $X$ is
    a closed set.
  \end{problem}
  \begin{solution}

  \end{solution}
  \clearpage

% --------------------------- Problem 4 ---------------------------- %
  \begin{problem}[5.5]
    Show that $\RR_{\rm LL}$ is normal.
  \end{problem}
  \begin{solution}

  \end{solution}
  \clearpage

% --------------------------- Problem 5 ---------------------------- %
  \begin{problem}[5.9]
    A topological space $X$ is normal if and only if for each closed
    set $A$ in $X$ and open set $U$ containing $A$ there exists an
    open set $V$ such that $A \subset V$, and $\ol V \subset U$.
  \end{problem}
  \begin{solution}

  \end{solution}

\end{document}