\documentclass{fkpset}

% \newgeometry{bmargin=1in, tmargin=1.25in, lmargin=.75in, rmargin=.75in}
% \fancyhfoffset[R]{.05cm}

\name{Forest Kobayashi}
\class{Math 147}
\duedate{03/13/2019}
\assignment{HW 5 Solutions}

\chead{HW 5 Solutions}
\rhead{Math 147 -- Spring, 2019}

\lfoot{Wednesday, March 13th 2019}

\renewcommand{\thesubfigure}{\roman{subfigure}}
\newcommand{\tstd}{\ensuremath \ms T_{\rm std}}
\let\lunateepsilon\epsilon
\renewcommand{\epsilon}{\varepsilon}
\newenvironment{why}{\begin{adjustwidth}{0.01\linewidth}{0.05\linewidth}~}%
  {\end{adjustwidth}}

\begin{document}
\pagestyle{plain}
\pagestyle{fancy}
  \vspace{-2.9cm}
  \begin{table}[H]
    \centering
    \begin{tabular}{@{}lcccccr@{}}\toprule
      Problems & 5.6(4) & 5.11 & 5.15(no normal) & 5.17 & 5.23 & Total \\ \midrule
      Points   &        &      &                 &      &      &       \\ \bottomrule
    \end{tabular}
  \end{table}
  \vspace{1cm}

% --------------------------- Problem 1 ---------------------------- %
  \begin{problem}[5.6(4)]
    Show that $\RR^2$ with the standard topology is normal.
  \end{problem}
  \begin{solution}
    First, we introduce some notation.
    \begin{leftbar}
      \textbf{Notational Note:} Let $(X, \ms T)$ be a topological
      space. Let $x \in X$, and let $Y \subset X$. Then define
      \[
        d(x, Y) = \inf_{y \in Y} d(x,y).
      \]
    \end{leftbar}

    \emph{Main Proof:} Let $A,B$ be disjoint closed subsets of
    $\RR^2$. For each $a\in A$, $b\in B$, let
    \begin{align*}
      \epsilon_a &= \frac{d(a,B)}{3} & \epsilon_b &= \frac{d(b,A)}{3}
    \end{align*}
    and note that by part (1), $\epsilon_a, \epsilon_b > 0$. Define
    \begin{align*}
      U &= \bigcup_{a \in A} B_{\epsilon_a}(a) & V &= \bigcup_{b \in B} B_{\epsilon_b}(b)
    \end{align*}
    and observe $U, V \in \tstd$, with $A \subset U$ and $B \subset
    V$. We want to show $U \cap V = \varnothing$.

    Suppose, to obtain a contradiction, that $U \cap V \neq
    \varnothing$. Let $x \in U \cap V$. Then there exist $a \in A$, $b
    \in B$ such that $x \in B_{\epsilon_a}(a) \cap B_{\epsilon_b}(b)$.
    It follows that
    \begin{align*}
      d(a,b)
      &\leq d(a,x) + d(x,b) \tag{$\star$}\\
      &\leq \epsilon_a + \epsilon_b
    \end{align*}
    WLOG, suppose $\epsilon_b \leq \epsilon_a$. Then
    \begin{align*}
      d(a,b)
      &\leq 2\epsilon_a \\
      &= \frac{2}{3} d(a,B)  \\
      &< d(a,b)
    \end{align*}
    a contradiction.\footnote{Note, here our contradiction is $d(a,b)
      < d(a,b)$. But we could also just skip ($\star$), and directly
      contradict the triangle inequality by $d(a,x) + d(x,b) <
      d(a,b)$. I prefer the former, just because it better matches
      arguments seen in Analysis.} Hence, $U \cap V = \varnothing$, so
    $U,V$ are disjoint open sets containing $A$ and $B$ respectively.
    Since $A,B$ were arbitrarily chosen, it follows that $\RR^2$ is
    normal, as desired.
    \begin{leftbar}
      \color{red} \textbf{Clarifying Note:} How did we get the final
      inequality? Well, note that
      \[
        d(a,B) = \inf_{b \in B} d(a,b)
      \]
      hence for all $b' \in B$, $d(a,B) \leq d(a,b')$. Since $d(a,B) >
      0$, it follows that for all $b' \in B$,
      \[
        \frac{2}{3}d(a,B) < d(a,b').
      \]
    \end{leftbar}
  \end{solution}
  \clearpage

% --------------------------- Problem 2 ---------------------------- %

  \begin{problem}[5.11 (The Incredible Shrinking Theorem)]
    A topological space $X$ is normal if and only if for each pair of
    open sets $U,V$ such that $U \cup V = X$, there exist open sets
    $U', V'$ such that $\ol{U'} \subset U$ and $\ol{V'} \subset V$,
    and $U' \cup V' = X$.
  \end{problem}
  \begin{solution}
    First, we prove a lemma.\footnote{I'm just proving it as a lemma
      so that I can offer two proofs. In an actual writeup, I'd just
      use one of them.} The $(\implies)$ direction will follow as a
    corollary.
    \begin{leftbar}\setcounter{section}{2}\vspace{-.25cm}
      % Man, I really ought to fix lemmas inside leftbars sometime
      \begin{lemma}
        Let $(X, \ms T)$ be normal. Let $U,V \in \ms T$ such that $U
        \cup V = X$. Then there exists $U' \in \ms T$ such that
        $\ol{U'} \subset U$, and $U' \cup V = X$.
      \end{lemma}

      I'll provide two proofs. The first uses theorem 5.9 (and is
      hence \emph{much} cleaner), while the second uses the definition
      of normality (and is hence much longer / more involed). I
      included both, so that people who tried to use normality could
      see how to proceed.

      \begin{adjustwidth}{.025\linewidth}{.025\linewidth}
        \color{RedOrange}
        \begin{boxedminipage}{\linewidth}
          \emph{Proof 1:} Note that $V^c$ is closed, and $V^c \subset
          U$. Then by theorem 5.9, there exists $U' \in \ms T$ such
          that
          \[
            V^c \subset U' \subset \ol{U'} \subset U.
          \]
          Note that $V^c \subset U' \implies U' \cup V = X$. Hence, we
          have our desired $U'$. \hfill $\square$
        \end{boxedminipage}
      \end{adjustwidth}
      ~
      \begin{adjustwidth}{.025\linewidth}{.025\linewidth}
        \color{TealBlue}
        \begin{boxedminipage}{\linewidth}
          \emph{Proof 2:} $U,V \in \ms T$ implies $U^c$ and $V^c$ are
          closed. Observe that
          \begin{align*}
            U^c \cap V^c
            &= (U \cup V)^c \\
            &= \varnothing,
          \end{align*}
          hence $U^c, V^c$ are disjoint closed sets. Then by
          definition of normality, there exist disjoint open sets
          $U',V'$ such that
          \[
            U^c \subset V' \qquad\quad \text{and} \quad\qquad V^c
            \subset U'.
          \]
          Note that $V^c \subset U' \implies U' \cup V = X$. It
          remains to show $\ol{U'} \subset U$. Since $U' \cap V' =
          \varnothing$ and $U^c \subset V'$, we have
          \[
            U' \subset V'^c \subset U
          \]
          and because $V'^c$ is closed, $\ol{U'} \subset V'^c$ as
          well. This proves the claim. \hfill $\square$
        \end{boxedminipage}
      \end{adjustwidth}
      Now, the main proof.
    \end{leftbar}

    \begin{iffproof}
    \item Suppose $X$ is normal.

        Let $U,V \in \ms T$ such that $U \cup V = X$. Then by the
        lemma, there exists $U' \in \ms T$ such that $\ol{U'} \subset
        U$, and $U' \cup V = X$. Now, applying the lemma to the pair
        $(V, U')$, we obtain the desired $V'$. \cmark

      \item Suppose that $\forall U,V \in \ms T \st U \cup V = X$,
        there exists $U', V' \st \ol{U'} \subset U$, $\ol{V'} \subset
        V$, and $U' \cup V' = X$. WTS $X$ is normal. We will apply
        Theorem 5.9.

        Let $A \subset X$ be an arbitrary closed set, and let $U \in
        \ms T$ such that $A \subset U$.\footnote{At least one such $U$
          exists, namely $X$, hence we can freely declare $U$ in this
          manner.} Observe that $A^c$ is open, and $U^c \subset A^c$.
        It follows that $X = U \cup A^c$. Then by hypothesis, there
        exists $U',V' \in \ms T$ such that
        \[
          \ol{U'} \subset U \qquad \qquad \qquad \ol{V'} \subset A^c
        \]
        and $U' \cup V' = X$. Observe that this last condition implies
        $(U')^c \subset V'$, hence we have
        \[
          (U')^c \subset V' \subset \ol{V'} \subset A^c
        \]
        Taking the complement and employing $U' \subset U$, we have
        \[
          A \subset (\ol{V'})^c \subset (V')^c \subset U' \subset U.
        \]
        Note that $(\ol{V'})^c$ is open, while $(V')^c$ is closed.
        Then by Theorem 3.20,
        \[
          \ol{(\ol{V'})^c} \subset (V')^c
        \]
        hence
        \[
          A \subset (\ol{V'})^c \subset \ol{(\ol{V'})^c} \subset U.
        \]
        Since $A$ and $U$ were arbitrarily chosen, Theorem 5.9 implies
        $X$ is normal.
    \end{iffproof}
  \end{solution}
  \clearpage

% --------------------------- Problem 3 ---------------------------- %
  \begin{problem}[5.15]
    Order topologies are $T_1$, Hausdorff, and regular.
  \end{problem}
  \begin{solution}
    Let $X$ be a totally ordered set, and $\ms T$ be the associated
    order topology. Denote the elements of the canonical basis as
    follows:
    \begin{itemize}
      \item $(-\infty, a) = \set{x \in X \MID x < a}$
      \item $(a, \infty) = \set{x \in X \MID a < x}$
      \item $(a,b) = \set{x \in X \MID a < x < b}$.
    \end{itemize}
    Square brackets will indicate inclusivity, as usual.
    \begin{note}
      \color{red} Although the notation here is almost identical to
      that of the standard topology on $\RR$, we need not have $X =
      \RR$. In fact, $X$ is guaranteed to have \emph{no algebraic
        structure} whatsoever. Be sure to keep this in mind as we
      proceed!
    \end{note}

    \begin{enumerate}[label=(\arabic*)]
      \item We apply Theorem 5.1. Let $x \in X$ be arbitrary. Then
        $(-\infty, x) \cup (x, \infty)$ is open. By complement,
        $\set{x}$ is closed, hence $(X,\ms T)$ is $T_1$.
      \item WLOG, suppose $x < y$. We proceed by casework.
        \begin{enumerate}[label=(\roman*)]
          \item Suppose there exists $z \in X$ such that $x < z < y$.
            Then $U = (-\infty, x)$, $V = (z, \infty)$ are disjoint
            open sets with $x \in U$, $y \in V$.
          \item Suppose no such $z$ exists. Then $U = (-\infty, y)$,
            $V = (x, \infty)$ are disjoint open sets with $x \in U$,
            $y \in V$.
        \end{enumerate}
        hence $(X, \ms T)$ is Hausdorff.
        \begin{remark}
          {\color{red} By Theorem 5.7.2, we actually just need to show
          regularity now that we have $T_1$. But in case you'd like to
          show Hausdorff constructively for extra practice, this is
          one way you might do it.

          Also, here's a graphic of tenuous worth to ``help''
          illustrate subcase (ii):}
          \begin{figure}[H]
            \centering
            \begin{tikzpicture}
              \fill[opacity=.2, blue, rounded corners=.75ex] (-1, -15pt) -- (5.9, -15pt) -- (5.9, -21pt) -- (-1, -21pt) -- cycle;
              \fill[opacity=.2, red, rounded corners=.75ex] (1, 15pt) -- (-5.9, 15pt) -- (-5.9, 21pt) -- (1, 21pt) -- cycle;

              \draw[latex-] (-6,0) -- (-1,0);
              \draw[-latex] (1,0) -- (6,0);
              \foreach \x in  {-5,-3,3,5}{
                \draw[shift={(\x,0)},color=black] (0pt,3pt) -- (0pt,-3pt);
              }

              \draw[shift={(-1,0)}, color=black] (0pt,3pt) -- (0pt,-3pt) node[below]
              {\color{red} $x$};
              \draw[shift={(1,0)}, color=black] (0pt,3pt) -- (0pt,-3pt) node[below]
              {\color{blue} $y$};

              \draw[{(-[}, dashed, thick, blue] (-1,-18pt) -- (1.075,-18pt);
              \draw[{-latex}, thick, blue] (1,-18pt) -- (6,-18pt);
              \draw[{(-[}, dashed, thick, red] (1,18pt) -- (-1.075,18pt);
              \draw[{-latex}, thick, red] (-1,18pt) -- (-6,18pt);

            \end{tikzpicture}
            \caption{Subcase (ii)}
          \end{figure}
          {\color{red} Note the gap between $x$ and $y$. $U$ is the
            top interval, $V$ is the bottom one.}
        \end{remark}
      \item To show regularity, we will employ Theorem 5.8. But first,
        a small Lemma.
        \begin{leftbar}
          \begin{lemma}
            Let $(a,b) \subset X$. Then $\ol{(a,b)} \subset \bk{a,b}$.
          \end{lemma}
          \emph{Proof:} Note that $X - \bk{a,b} = (-\infty, a) \cup
          (b,\infty)$ is open, hence $[a,b]$ is closed. By Theorem
          3.20, we have $\ol{(a,b)} \subset [a,b]$.
        \end{leftbar}

        Let $x \in X$ be arbitrary. Let $U \in \ms T$ such that $x \in
        U$. Then there exists $\set{B_\alpha}_{\alpha \in \lambda}$
        such that
        \[
          U = \bigcup_{\alpha \in \lambda} B_\alpha.
        \]
        Let $\alpha_0 \in \lambda \st x \in B_{\alpha_0}$. Then there
        exists $a,b \in X\cup \set{-\infty, \infty}$ such that
        \[
          x \in (a,b) \subset B_{\alpha_0}.
        \]
        We now have four subcases.
        \begin{enumerate}[label=\roman*)]
          \item Suppose there exist no $a',b' \in (a,b)$ such that $x
            \in (a', b') \subset (a,b)$. Then
        \end{enumerate}
    \end{enumerate}
  \end{solution}
  \clearpage

% --------------------------- Problem 4 ---------------------------- %
  \begin{problem}[5.17]
    Let $X$ and $Y$ be regular. Then $X \times Y$ is regular.
  \end{problem}
  \begin{solution}
  \end{solution}
  \clearpage

% --------------------------- Problem 5 ---------------------------- %
  \begin{problem}[5.23]
    Let $A$ be a closed subset of a normal space $X$. Then $A$ is
    normal when given the relative topology.
  \end{problem}
  \begin{solution}
    Let $\ms T_X$ be the topology on $X$, and $\ms C_X$ be the set of
    closed sets in $(X,\ms T)$. Similarly, let $\ms T_A, \ms C_A$.

    Let $B,C \in \ms C_A$ be disjoint. Then by Theorem 4.28, there
    exist $B', C' \in \ms C_X$ such that
    \[
      B = B' \cap A \qquad \qquad \qquad C = C' \cap A.
    \]
    Then since $\ms C_X$ is closed under arbitrary intersection, it
    follows that $B, C$ are closed in $(X, \ms T)$.\footnote{OK, I
      know I defined $\ms C_X$ above, but I was worried that all the
      script $C$'s flying around were getting confusing!} Hence by
    normality, there exist disjoint $U,V \in \ms T_X$ such that $B
    \subset U$, and $C \subset V$. Note that
    \[
      B = (B \cap A) \subset U \cap A \qquad \qquad \qquad C = (C \cap
      A) \subset V \cap A,
    \]
    and $U \cap A$, $V \cap A$ are open sets in $(A, \ms T_A)$. Since
    they are also disjoint, we see $B,C$ are separated by disjoint
    open sets in $(A, \ms T_A)$.

    Finally, since $B,C$ were arbitrarily chosen, it follows that $A$
    is normal with the relative topology.
  \end{solution}

\end{document}