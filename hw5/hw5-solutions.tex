\documentclass{fkpset}

% \newgeometry{bmargin=1in, tmargin=1.25in, lmargin=.75in, rmargin=.75in}
% \fancyhfoffset[R]{.05cm}

\name{Forest Kobayashi}
\class{Math 147}
\duedate{03/13/2019}
\assignment{HW 5 Solutions}

\chead{HW 5 Solutions}
\rhead{Math 147 -- Spring, 2019}

\lfoot{Wednesday, March 13th 2019}

\renewcommand{\thesubfigure}{\roman{subfigure}}
\newcommand{\tstd}{\ensuremath \ms T_{\rm std}}
\let\lunateepsilon\epsilon
\renewcommand{\epsilon}{\varepsilon}
\newenvironment{why}{\begin{adjustwidth}{0.01\linewidth}{0.05\linewidth}~}%
  {\end{adjustwidth}}

\begin{document}
\pagestyle{plain}
\pagestyle{fancy}
  \vspace{-2.9cm}
  \begin{table}[H]
    \centering
    \begin{tabular}{@{}lcccccr@{}}\toprule
      Problems & 5.6(4) & 5.11 & 5.15(no normal) & 5.17 & 5.23 & Total \\ \midrule
      Points   &        &      &                 &      &      &       \\ \bottomrule
    \end{tabular}
  \end{table}
  \vspace{1cm}

% --------------------------- Problem 1 ---------------------------- %
  \begin{problem}[5.6(4)]
    Show that $\RR^2$ with the standard topology is normal.
  \end{problem}
  \begin{solution}
    First, we introduce some notation.
    \begin{leftbar}
      \textbf{Notational Note:} Let $(X, \ms T)$ be a topological
      space. Let $x \in X$, and let $Y \subset X$. Then define
      \[
        d(x, Y) = \inf_{y \in Y} d(x,y)
      \]
      now the main proof.
    \end{leftbar}

    \emph{Main Proof:} Let $A,B$ be disjoint closed subsets of
    $\RR^2$. For each $a\in A$, $b\in B$, let
    \begin{align*}
      \epsilon_a &= \frac{d(a,B)}{3} & \epsilon_b &= \frac{d(b,A)}{3}
    \end{align*}
    and note that by part (1), $\epsilon_a, \epsilon_b > 0$. Define
    \begin{align*}
      U &= \bigcup_{a \in A} B_{\epsilon_a}(a) & V &= \bigcup_{b \in B} B_{\epsilon_b}(b)
    \end{align*}
    and observe $U, V \in \tstd$, with $A \subset U$ and $B \subset
    V$.

    Suppose, to obtain a contradiction, that $U \cap V \neq
    \varnothing$. Let $x \in U \cap V$. Then there exist $a \in A$, $b
    \in B$ such that $x \in B_{\epsilon_a}(a) \cap B_{\epsilon_b}(b)$.
    It follows that
    \begin{align*}
      d(a,b)
      &\leq d(a,x) + d(x,b) \\
      &\leq \epsilon_a + \epsilon_b
    \end{align*}
    WLOG, suppose $\epsilon_b \leq \epsilon_a$. Then
    \begin{align*}
      d(a,b)
      &\leq 2\epsilon_a \\
      &= \frac{2}{3} d(a,B)
    \end{align*}
    a contradiction. Hence $U \cap V = \varnothing$, so $\RR^2$ is
    normal, as desired.
    \begin{leftbar}
      \color{red} \textbf{Clarifying Note:} Why is this a
      contradiction? Because
      \begin{align*}
        d(a,B) = \inf_{b \in B} d(a,b)
      \end{align*}
      hence for all $b' \in B$, $d(a,B) \leq d(a,b')$. Thus our result
      would imply
      \[
        d(a,b) \leq \frac{2}{3} d(a,b),
      \]
      which holds iff $d(a,b) = 0$, a contradiction.
    \end{leftbar}
  \end{solution}
  \clearpage

% --------------------------- Problem 2 ---------------------------- %

  \begin{problem}[5.11 (The Incredible Shrinking Theorem)]
    A topological space $X$ is normal if and only if for each pair of
    open sets $U,V$ such that $U \cup V = X$, there exist open sets
    $U', V'$ such that $\ol{U'} \subset U$ and $\ol{V'} \subset V$,
    and $U' \cup V' = X$.
  \end{problem}
  \begin{solution}
  \end{solution}
  \clearpage

% --------------------------- Problem 3 ---------------------------- %
  \begin{problem}[5.15]
    A space $(X, \ms T)$ is $T_1$ if and only if every point in $X$ is
    a closed set.
  \end{problem}
  \begin{solution}
    Order topologies are $T_1$, Hausdorff, and regular.
  \end{solution}
  \clearpage

% --------------------------- Problem 4 ---------------------------- %
  \begin{problem}[5.17]
    Let $X$ and $Y$ be regular. Then $X \times Y$ is regular.
  \end{problem}
  \begin{solution}
  \end{solution}
  \clearpage

% --------------------------- Problem 5 ---------------------------- %
  \begin{problem}[5.23]
    Let $A$ be a closed subset of a normal space $X$. Then $A$ is
    normal when given the relative topology.
  \end{problem}
  \begin{solution}
  \end{solution}

\end{document}