\documentclass{fkpset}

% \newgeometry{bmargin=1in, tmargin=1.25in, lmargin=.75in, rmargin=.75in}
% \fancyhfoffset[R]{.05cm}

\name{Forest Kobayashi}
\class{Math 147}
\duedate{03/13/2019}
\assignment{HW 5 Solutions}

\chead{HW 5 Sample Solutions}
\rhead{Math 147 -- Spring, 2019}

\lfoot{Wednesday, March 13th 2019}

\problems{5.6(4), 5.11, 5.15(no normal), 5.17, 5.23}

\usepackage{hyperref}

\newcommand{\tstd}{\ensuremath \ms T_{\rm std}}
\newcommand{\tprod}{\ensuremath \ms T_{\rm prod}}

\begin{document}
\pagestyle{plain}
\pagestyle{fancy}
  % \vspace{-2.9cm}
  \pointstable{}

  \vspace{1cm}

% --------------------------- Problem 1 ---------------------------- %
  \begin{problem}[5.6(4)]
    Show that $\RR^2$ with the standard topology is normal.
  \end{problem}
  \begin{solution}
    First, we introduce some notation.
    \begin{leftbar}
      \textbf{Notational Note:} Let $(X, \ms T)$ be a topological
      space. Let $x \in X$, and let $Y \subset X$. Then define
      \[
        d(x, Y) = \inf_{y \in Y} d(x,y).
      \]
    \end{leftbar}

    \emph{Main Proof:} Let $A,B$ be disjoint closed subsets of
    $\RR^2$. For each $a\in A$, $b\in B$, let
    \begin{align*}
      \epsilon_a &= \frac{d(a,B)}{2} & \epsilon_b &= \frac{d(b,A)}{2}
    \end{align*}
    and note that by part (1), $\epsilon_a, \epsilon_b > 0$. Define
    \begin{align*}
      U &= \bigcup_{a \in A} B_{\epsilon_a}(a) & V &= \bigcup_{b \in B} B_{\epsilon_b}(b)
    \end{align*}
    and observe $U, V \in \tstd$, with $A \subset U$ and $B \subset
    V$. We want to show $U \cap V = \varnothing$.

    Suppose, to obtain a contradiction, that $U \cap V \neq
    \varnothing$. Let $x \in U \cap V$. Then there exist $a \in A$, $b
    \in B$ such that $x \in B_{\epsilon_a}(a) \cap B_{\epsilon_b}(b)$.
    It follows that
    \begin{align*}
      d(a,b)
      &\leq d(a,x) + d(x,b) \tag{$\star$}\\
      &< \epsilon_a + \epsilon_b
    \end{align*}
    WLOG, suppose $\epsilon_b \leq \epsilon_a$. Then
    \begin{align*}
      d(a,b)
      &< 2\epsilon_a \\
      &= d(a,B)  \\
      &\leq d(a,b)
    \end{align*}
    so $d(a,b) < d(a,b)$, a contradiction.\footnote{The align
      environment given should be read as ``$d(a,b) < 2\epsilon_a =
      d(a,B) \leq d(a,b)$,'' not as $d(a,b) < 2\epsilon_a$, $d(a,b) =
      d(a,B)$, and so on. Also, here our contradiction is $d(a,b) <
      d(a,b)$, but we could also just skip ($\star$) and directly
      contradict the triangle inequality by $d(a,x) + d(x,b) <
      d(a,b)$. I prefer the former, just because it better matches
      arguments seen in Analysis.}
    Hence, $U \cap V = \varnothing$, so $U,V$ are disjoint open sets
    containing $A$ and $B$ respectively. Since $A,B$ were arbitrarily
    chosen, it follows that $\RR^2$ is normal, as desired.
  \end{solution}
  \clearpage

% --------------------------- Problem 2 ---------------------------- %

  \begin{problem}[5.11 (The Incredible Shrinking Theorem)]
    A topological space $X$ is normal if and only if for each pair of
    open sets $U,V$ such that $U \cup V = X$, there exist open sets
    $U', V'$ such that $\ol{U'} \subset U$ and $\ol{V'} \subset V$,
    and $U' \cup V' = X$.
  \end{problem}
  \quad\
  I'll provide two solutions: one using Theorem 5.9, another using
  Theorem 5.10.
  \begin{solution}[Solution 1:]
    First, we prove a lemma.\footnote{I'm just proving it as a lemma
      so that I can offer two proofs. In an actual writeup, I'd just
      use one of them.} The $(\implies)$ direction will follow as a
    corollary.
    \begin{leftbar}\setcounter{section}{2}\vspace{-.25cm}
      % Man, I really ought to fix lemmas inside leftbars sometime
      \begin{lemma}
        Let $(X, \ms T)$ be normal. Let $U,V \in \ms T$ such that $U
        \cup V = X$. Then there exists $U' \in \ms T$ such that
        $\ol{U'} \subset U$, and $U' \cup V = X$.
      \end{lemma}

      I'll provide two proofs. The first uses theorem 5.9 (and is
      hence \emph{much} cleaner), while the second uses the definition
      of normality (and is hence much longer / more involed). I
      included both, so that people who tried to use normality
      directly could see how to proceed.

      \begin{adjustwidth}{.025\linewidth}{.025\linewidth}
        \color{RedOrange}
        \begin{boxedminipage}{\linewidth}
          \emph{Proof 1:} Note that $V^c$ is closed, and $V^c \subset
          U$. Then by theorem 5.9, there exists $U' \in \ms T$ such
          that
          \[
            V^c \subset U' \subset \ol{U'} \subset U.
          \]
          Note that $V^c \subset U' \implies U' \cup V = X$. Hence, we
          have our desired $U'$. \hfill $\square$
        \end{boxedminipage}
      \end{adjustwidth}
      ~
      \begin{adjustwidth}{.025\linewidth}{.025\linewidth}
        \color{TealBlue}
        \begin{boxedminipage}{\linewidth}
          \emph{Proof 2:} $U,V \in \ms T$ implies $U^c$ and $V^c$ are
          closed. Observe that
          \begin{align*}
            U^c \cap V^c
            &= (U \cup V)^c \\
            &= \varnothing,
          \end{align*}
          hence $U^c, V^c$ are disjoint closed sets. Then by
          definition of normality, there exist disjoint open sets
          $U',V'$ such that
          \[
            U^c \subset V' \qquad\quad \text{and} \quad\qquad V^c
            \subset U'.
          \]
          Note that $V^c \subset U' \implies U' \cup V = X$. It
          remains to show $\ol{U'} \subset U$. Since $U' \cap V' =
          \varnothing$ and $U^c \subset V'$, we have
          \[
            U' \subset V'^c \subset U
          \]
          and because $V'^c$ is closed, $\ol{U'} \subset V'^c$ as
          well. This proves the claim. \hfill $\square$
        \end{boxedminipage}
      \end{adjustwidth}
      Now, the main proof.
    \end{leftbar}

    \begin{iffproof}
    \item Suppose $X$ is normal.

        Let $U,V \in \ms T$ such that $U \cup V = X$. Then by the
        lemma, there exists $U' \in \ms T$ such that $\ol{U'} \subset
        U$, and $U' \cup V = X$. Now, applying the lemma to the pair
        $(V, U')$, we obtain the desired $V'$. \cmark

      \item Suppose that $\forall U,V \in \ms T \st U \cup V = X$,
        there exists $U', V' \in \ms T \st \ol{U'} \subset U$,
        $\ol{V'} \subset V$, and $U' \cup V' = X$. WTS $X$ is normal.
        We will apply Theorem 5.9.

        Let $A \subset X$ be an arbitrary closed set, and let $U \in
        \ms T$ such that $A \subset U$.\footnote{At least one such $U$
          exists, namely $X$, hence we can freely declare $U$ in this
          manner.} Observe that $A^c$ is open, and $U^c \subset A^c$.
        It follows that $X = U \cup A^c$. Then by hypothesis, there
        exists $U',V' \in \ms T$ such that
        \[
          \ol{U'} \subset U \qquad \qquad \qquad \ol{V'} \subset A^c
        \]
        and $U' \cup V' = X$. From this it follows that $(U')^c
        \subset V'$, hence
        \[
          (U')^c \subset V' \subset A^c.
        \]
        Taking the complement, we have
        \[
          A \subset (V')^c \subset U',
        \]
        and since $\ol{U'} \subset U$, this yields
        \[
          A \subset U' \subset \ol{U'} \subset U
        \]
        as desired. Since $A$ and $U$ were arbitrarily chosen, Theorem
        5.9 implies $X$ is normal. \cmark
    \end{iffproof}
  \end{solution}
  \hrulefill
  \begin{solution}[Solution 2:]
    We employ Theorem 5.10.
    \begin{iffproof}
      \item Suppose $X$ is normal. Let $U,V \in \ms T$ such that $U
        \cup V = X$. Then $U^c, V^c$ are closed, and by DeMorgan's
        Laws,
        \[
          U^c \cap V^c = (U \cup V)^c = \varnothing,
        \]
        hence they are disjoint as well. Then by Theorem 5.10, there
        exist disjoint $U_0, V_0 \in \ms T$ such that
        \[
          U^c \subset U_0\qquad\qquad V^c \subset V_0 \qquad\qquad
          \ol{U_0} \cap \ol{V_0} = \varnothing.
        \]
        Because $U_0 \subset \ol{U_0}$ and $V_0 \subset \ol{V_0}$,
        taking complements yields
        \[
          \pn{\ol{U_0}}^c \subset (U_0)^c \subset U \qquad\qquad
          \pn{\ol{V_0}}^c \subset (V_0)^c \subset V \qquad\qquad
          \pn{\ol{U_0}}^c \cup \pn{\ol{V_0}}^c
          = X.
        \]
        Let $U' = \pn{\ol{U_0}}^c$ and $V' =
        \pn{\ol{V_0}}^c$, and note that these are open. Then
        the above can be reexpressed as
        \[
          U' \subset (U_0)^c \subset U \qquad\qquad V' \subset (V_0)^c
          \subset V \qquad\qquad U' \cup V' = X,
        \]
        and since $(U_0)^c,\ (V_0)^c$ are closed, Theorem 3.20 implies
        \[
          U' \subset \ol{U'} \subset (U_0)^c \subset U \qquad\qquad
          V' \subset \ol{V'} \subset (V_0)^c \subset V
        \]
        as desired. \cmark
      \item Suppose $\forall U,V \in \ms T \st U\cup V = X$, there
        exists $U', V' \in \ms T \st \ol{U'} \subset U,\ \ol{V'}
        \subset V$, and $U' \cup V' = X$. WTS $X$ is normal.

        Let $A,B$ be arbitrary disjoint closed sets. Then $U=A^c$,
        $V=B^c$ are open, and $U\cup V = X$ (DeMorgan's Laws).

        By hypothesis, there exists $U',V' \in \ms T$ such that
        \[
          \ol{U'} \subset U \qquad \qquad \ol{V'} \subset V \qquad
          \qquad U' \cup V' = X.
        \]
        Since $U' \subset \ol{U'}$ and $V' \subset \ol{V'}$,
        complementation yields
        \[% wao I love manual horizontal spacing
          U^c \subset \pn{\ol{U'}}^c \subset (U')^c
          \qquad\qquad V^c \subset \pn{\ol{V'}}^c \subset
          (V')^c \qquad\qquad (U')^c \cap (V')^c
          = \varnothing.
        \]
        Finally, substituting $U^c = A$ and $V^c = B$, we see
        $\pn{\ol{U'}}^c$, $\pn{\ol{V'}}^c$ are disjoint open
        sets separating $A$ and $B$. Thus $X$ is normal, as desired.
        \cmark
      \end{iffproof}
  \end{solution}
  \clearpage

% --------------------------- Problem 3 ---------------------------- %
  \begin{problem}[5.15]
    Order topologies are $T_1$, Hausdorff, and regular.
  \end{problem}
  \begin{solution}
    Let $X$ be a totally ordered set, and $\ms T$ be the associated
    order topology. Denote the elements of the canonical basis as
    follows:
    \begin{itemize}
      \item $(-\infty, a) = \set{x \in X \MID x < a}$
      \item $(a, \infty) = \set{x \in X \MID a < x}$
      \item $(a,b) = \set{x \in X \MID a < x < b}$.
    \end{itemize}
    Square brackets will indicate inclusivity, as usual.
    \begin{note}
      \color{red} Although the notation here is almost identical to
      that of the standard topology on $\RR$, we need not have $X =
      \RR$. In fact, $X$ is guaranteed to have \emph{no algebraic
        structure} whatsoever. Be sure to keep this in mind as we
      proceed!
    \end{note}

    \begin{enumerate}[label=(\arabic*)]
      \item We apply Theorem 5.1. Let $x \in X$ be arbitrary. Then
        $(-\infty, x) \cup (x, \infty)$ is open. By complement,
        $\set{x}$ is closed, hence $(X,\ms T)$ is $T_1$.
        \begin{remark}
          {\color{red} By Theorem 5.7.2, we actually just need to show
            regularity now that we have $T_1$. But in case you'd like
            to show Hausdorff constructively for extra practice, I've
            included a proof of Hausdorffness below.}
        \end{remark}
      \item WLOG, suppose $x < y$. We proceed by casework.
        \begin{enumerate}[label=(\roman*)]
          \item Suppose $(x,y) \neq \varnothing$. Let $z \in (x,y)$.
            Then $U = (-\infty, z)$, $V = (z, \infty)$ are disjoint
            open sets with $x \in U$, $y \in V$.
          \item Suppose $(x,y) = \varnothing$. Then $U = (-\infty,
            y)$, $V = (x, \infty)$ are disjoint open sets with $x \in
            U$, $y \in V$.
        \end{enumerate}
        hence $(X, \ms T)$ is Hausdorff.
        \begin{figure}[H]
          \centering
          \begin{tikzpicture}
            \fill[opacity=.2, blue, rounded corners=.75ex] (-1, -15pt) -- (5.9, -15pt) -- (5.9, -21pt) -- (-1, -21pt) -- cycle;
            \fill[opacity=.2, red, rounded corners=.75ex] (1, 15pt) -- (-5.9, 15pt) -- (-5.9, 21pt) -- (1, 21pt) -- cycle;

            \draw[latex-] (-6,0) -- (-1,0);
            \draw[-latex] (1,0) -- (6,0);
            \foreach \x in  {-5,-3,3,5}{
              \draw[shift={(\x,0)},color=black] (0pt,3pt) -- (0pt,-3pt);
            }

            \draw[shift={(-1,0)}, color=black] (0pt,3pt) -- (0pt,-3pt) node[below]
            {\color{red} $x$};
            \draw[shift={(1,0)}, color=black] (0pt,3pt) -- (0pt,-3pt) node[below]
            {\color{blue} $y$};

            \draw[{(-[}, dashed, thick, blue] (-1,-18pt) -- (1.075,-18pt) node[at start, left] {$V$};
            \draw[{-latex}, thick, blue] (1,-18pt) -- (6,-18pt);
            \draw[{(-[}, dashed, thick, red] (1,18pt) -- (-1.075,18pt) node[at start, right] {$U$};
            \draw[{-latex}, thick, red] (-1,18pt) -- (-6,18pt);

          \end{tikzpicture}
          \caption{Subcase (ii). Note the gap between $x$ and $y$.}
        \end{figure}
      \item To show regularity, we will employ Theorem 5.8. But first,
        a small Lemma.
        \begin{leftbar}
          \begin{lemma}
            Let $(a,b) \subset X$. Then $\ol{(a,b)} \subset \bk{a,b}$.
          \end{lemma}
          \emph{Proof:} Note that $X - \bk{a,b} = (-\infty, a) \cup
          (b,\infty)$ is open, hence $[a,b]$ is closed. By Theorem
          3.20, we have $\ol{(a,b)} \subset [a,b]$.
        \end{leftbar}
        \begin{remark} \color{red}
          We actually can't do better than this in the general case
          (i.e., we need not have $\ol{(a,b)} = [a,b]$). For example,
          you can find subspaces of the lexicographically ordered
          square that refuse to play nice. Also, if $X$ is a discrete
          set (such as $\NN$ or $\ZZ$), plenty of counterexamples
          exist.
        \end{remark}
        Let $x \in X$ be arbitrary, and let $U \in \ms T$ such that $x
        \in U$. Then there exist $a,b \in X\cup \set{-\infty, \infty}$
        such that
        \[
          x \in (a,b) \subset U.\footnote{Note, this is just a concise
          way of declaring a basic open set.}
        \]
        \textbf{Claim:} There exists $(a',b') \subset (a,b)$ such that
        $x \in (a',b') \subset \ol{(a',b')} \subset (a,b)$.

        \textbf{Proof of Claim:} {\color{red} When typing this up, I
          found a slightly cleaner version of the argument I was using
          at \url{http://web.math.ku.dk/~moller/e02/3gt/opg/S31.pdf},
          and have modified my proof accordingly.}

        Let $A = (a, x)$, and $B = (x, b)$. Then we have four
        subcases.
        \begin{enumerate}[label=\roman*)]
          \item Suppose that $A,B = \varnothing$. Then $(a,b) =
            \set{x}$, which is clopen. Hence take $(a',b') = (a,b)$,
            and the claim holds. \cmark
          \item Suppose $A = \varnothing$ and $B \neq \varnothing$,
            and let $b' \in B$. Then let $a'=a$, and note $(a',b') =
            [x, b')$. Hence $x \in (a',b')$, and
            \[
              \ol{(a', b')} = \ol{[x, b')} \subset [x,b'] \subset
              (a,b)
            \]
            so the claim holds. \cmark
          \item Supposet $A \neq \varnothing$ and $B = \varnothing$.
            Analogously to the above, we let $a' \in A$ and $b'=b$,
            which yields
            \[
              \ol{(a',b')} = \ol{(a',x]} \subset [a',x] \subset (a,b)
            \]
            as desired. \cmark
          \item Suppose $A \neq \varnothing \neq B$. Then let $a' \in
            A$, $b' \in B$. It follows that
            \[
              x \in (a',b') \subset \ol{(a',b')} \subset [a',b']
              \subset (a,b)
            \]
            as desired. \cmark
        \end{enumerate}
        since these cases are exhaustive, this proves the claim. Then
        by Theorem 5.8, $X$ is regular, as desired.
    \end{enumerate}
  \end{solution}
  \clearpage

% --------------------------- Problem 4 ---------------------------- %
  \begin{problem}[5.17]
    Let $X$ and $Y$ be regular. Then $X \times Y$ is regular.
  \end{problem}
  \begin{solution}
    We prove a lemma.
    \begin{leftbar}
      \begin{lemma}
        Let $A \subset X$, $B \subset Y$. Then $\ol{A \times B} =
        \ol{A} \times \ol{B}$.
      \end{lemma}
      We use the notation $\mc L(S)$ to denote the limit points of a
      set $S$. Here're a few proofs:
      ~
      \begin{adjustwidth}{.025\linewidth}{.025\linewidth}
        \color{RedOrange}
        \begin{boxedminipage}{\linewidth}
          \emph{Proof 1:} The claim is equivalent to
          \[
            p \in \ol{A \times B} \iff p \in \ol{A} \times \ol{B}.
          \]
          We proceed by contrapositive. That is,
          \[
            p \not\in \ol{A \times B} \iff p \not \in \ol{A} \times
            \ol{B}.\footnote{Note, this is just saying that the set
              complements are equal.}
          \]
          We prove both directions simultaneously.\footnote{This
            introduces a mess with variable quantifications, but
            hopefully the argument makes sense} The following are
          equivalent:
          \begin{enumerate}[label=(\arabic*)]
            \item $p = (p_x, p_y) \not\in \ol{A\times B}$
            \item There exists $U \in \ms \tprod \st p \in U$ and $(U
              - \set{p}) \cap (A \times B) = \varnothing$
            \item For $U$ quantified as above, there exists $B = U_A
              \times U_B \in \ms B_{\rm prod}$ such that $p \in B$,
              and
              \begin{align*}
                \varnothing
                &= B \cap (A \times B) \\
                &= (U_A \times U_B) \cap (A \times B) \\
                &= (U_A \cap A) \times (U_B \cap B).
              \end{align*}
            \item There exists $U_A \in \ms T_A$, $U_B \in \ms T_B$
              such that at least one of $(U_A \cap A),\ (U_B \cap B)$
              is empty
            \item At least one of $p_x \not \in \ol{A},\ p_y \not \in
              \ol{B}$ is true
            \item $p \not \in \ol{A} \times \ol{B}$
          \end{enumerate}
          \hfill $\square$
        \end{boxedminipage}
      \end{adjustwidth}
      \begin{adjustwidth}{.025\linewidth}{.025\linewidth}
        \color{TealBlue}
        \begin{boxedminipage}{\linewidth}
          \emph{Proof 2:} We prove the claim directly.
          \begin{seteqproof}
            \item Let $p = (p_A, p_B) \in \ol{A \times B}$. We want to
              show $p_A \in \ol{A}$ or $p_B \in \ol{B}$.\footnote{The
                or here is inclusive.}
              \begin{enumerate}[label=(\arabic*)]
                \item Suppose $p \in A \times B$. Then we're done.
                  \cmark
                \item Now, suppose $p \in \mc L(A \times B).$ Let $U_A
                  \in \ms T_A$ and $U_B \in \ms T_B$ be arbitrarily
                  chosen. Then $U = U_A \times U_B \in \tprod$, and
                  hence
                  \[
                    (U - \set{p}) \cap (A \times B) \neq \varnothing
                  \]
                  Thus, let $q = (q_A, q_B) \in (U - \set{p}) \cap (A
                  \times B)$. It follows that
                  \[
                    \pi_x(q) \in (\pi_x(U - \set{p}) \cap \pi_x(A
                    \times B))
                  \]
                  or equivalently,
                  \[
                    q_A \in (U_A - \set{p_A}) \cap A
                  \]
                  and similarly, $q_B \in (U_B - \set{p_B}) \cap B$.
                  Hence, $\ol{A \times B} \subset \ol{A} \times
                  \ol{B}$. \cmark
              \end{enumerate}
            \item The reverse direction essentially consists of
              reversing the steps above. Note, you need to consider
              both of the cases $p \in \mc L(A) \times B$ and $p \in A
              \times \mc L(B)$.
          \end{seteqproof}
          \hfill $\square$
        \end{boxedminipage}
      \end{adjustwidth}
      \begin{adjustwidth}{.025\linewidth}{.025\linewidth}
        \color{Magenta}
        \begin{boxedminipage}{\linewidth}
          \emph{Proof 3:} We prove the two directions separately.
          \begin{seteqproof}
            \item The product of closed sets is closed.\footnote{As
                justification, note that $(\ol{A})^c \times
                (\ol{B})^c$ is a product of open sets, and is thus
                open in $\tprod$, hence $\ol{A} \times \ol{B}$ is
                closed.} Since $A \times B \subset \ol{A} \times
              \ol{B}$, Theorem 3.20 implies
              \[
                \ol{A \times B} \subset \ol{\ol{A} \times \ol{B}} =
                \ol{A} \times \ol{B}.
              \]
            \item Use either of the arguments above.
          \end{seteqproof}
          \hfill $\square$
        \end{boxedminipage}
      \end{adjustwidth}
    \end{leftbar}
    We proceed by Theorem 5.8.

    Let $p \in X \times Y$ be arbitrary, and let $U \in \ms T_{\rm
      prod}$ such that $p \in U$. Then by definition of $\ms T_{\rm
      prod}$, there exist $U_X \in \ms T_X$, $U_Y \in \ms T_Y$ such
    that
    \[
      p \in U_X \times U_Y \subset U.
    \]
    Since $X,Y$ are regular, there exist $V_X \in \ms T_X$, $V_Y \in
    \ms T_Y$ such that
    \[
      \pi_x(p) \in V_X \subset \ol{V_X} \subset B_X \qquad\quad
      \text{and} \quad\qquad \pi_y(p) \in V_Y \subset \ol{V_Y} \subset
      B_Y.
    \]
    Thus $p \in V_X \times V_Y$, which is open in $\tprod$. Then
    \[
      p \in V_X \times V_Y \subset \ol{V_X \times V_Y} = \ol{V_X}
      \times \ol{V_Y} \subset B_X \times B_Y \subset U
    \]
    so by Theorem 5.8, $X \times Y$ is regular.
  \end{solution}
  \clearpage

% --------------------------- Problem 5 ---------------------------- %
  \begin{problem}[5.23]
    Let $A$ be a closed subset of a normal space $X$. Then $A$ is
    normal when given the relative topology.
  \end{problem}
  \begin{solution}
    Let $\ms T_X$ be the topology on $X$, and $\ms C_X$ be the set of
    closed sets in $(X,\ms T)$. Define $\ms T_A, \ms C_A$ analogously
    for the relative topology on $A$.

    Let $B,C \in \ms C_A$ be disjoint. Then by Theorem 4.28, there
    exist $B', C' \in \ms C_X$ such that
    \[
      B = B' \cap A \qquad \qquad \qquad C = C' \cap A.
    \]
    Then since $\ms C_X$ is closed under arbitrary intersection, it
    follows that $B, C$ are closed in $(X, \ms T)$ as
    well.\footnote{OK, I know I defined $\ms C_X$ above, but I was
      worried that all the script $C$'s flying around were getting
      confusing!} Then by normality, there exist disjoint $U,V \in \ms
    T_X$ such that $B \subset U$ and $C \subset V$. Observe
    \[
      B = (B \cap A) \subset (U \cap A) \qquad \qquad \qquad C = (C
      \cap A) \subset (V \cap A),
    \]
    and by definition, $(U \cap A)$, $(V \cap A)$ are open sets in
    $(A, \ms T_A)$. Since $U \cap V = \varnothing$, we have
    \[
      (U\cap A) \cap (V\cap A)= \varnothing
    \]
    as well, thus we have found disjoint open sets separating $A,B$.
    Hence, $A$ is normal with the relative topology, as desired.
  \end{solution}

\end{document}